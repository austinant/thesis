\chapter{Monoids and Factorizations} \label{ch:fundamentals}
%%%%%%
%%%%%%


\section{Fundamentals of Factorization Theory} \label{subsec:factorizations}
We begin by laying out the framework in which we will study factorizations.

\begin{defn} \label{defn:atom}
	Let $H$ be a monoid.
	\begin{itemize}
		\item $u\in H$ is a \textbf{unit} if there is $v\in H$ with $uv = vu = 1$.  
		The group of units in $H$ is denoted by $H^\times$.  
		$H$ is called \textit{reduced} if $H^\times = \{1\}$.
		\item $x,y\in H$ are \textbf{associates} if there are units $u,v\in H^\times$ so that $x = uyv$. In this case, we write $x \sim y$.
		\item $a\in H\setminus H^\times$ is an \textbf{atom} if, whenever $a = xy$, either $x\in H^\times$ or $y\in H^\times$. The set of atoms of $H$ is denoted by $\A(H)$.
		\item $x\in H$ is \textbf{idompotent} (or \textbf{an idempotent}) if $x^2 = x$.
	\end{itemize}
\end{defn}

The existing body of work on factorization theory contains several reasonable definitions of ``associate'' and ``atom'' which are not always equivalent.
Here, we have chosen one such set of definitions; this choice has consequences on the definitions to be laid out in the remainder of this section, and on the nature of the results one may prove.
Fortunately, previous entries in the literature have taken care to compare some of these alternative notions.
Look to Section \ref{sub:other-factorizations} for more a more detailed digression on different definitions of ``associate'' and ``atom'' and how they relate to one another.

\begin{defn} \label{def:divisibility}
	Let $H$ be a monoid and let $x,y\in H$.
	We say $x$ \textbf{divides} $y$ in $H$ (written $x|_H y$, or $x|y$ if the monoid is clear from context) if there are $y,z\in H$ with $zxw = y$.
\end{defn}

We will use the language of free monoids heavily throughout as a convenient way of precisely describing information about factorizations of elements.  
This method of bookkeeping is borrowed from \cite{fan-tringali18}, which is in turn based on the usage of free abelian monoids in \cite{geroldinger-hk06} and much of the subsequent literature on factorization theory as studied from the monoid point of view.

\begin{defn} \label{def:free monoid}
Let $S$ be a set.
The \textbf{free monoid} on $S$ is the set
\[\F^*(S) := \{s_1*\cdots*s_\ell: \ell\in \NN \textrm{ and } s_i\in S \textrm{ for each } i\in [1,\ell] \} \]
of formal words whose letters belong to $S$.
Its operation, denoted by $*$, is called \textbf{concatenation}.

Let $\mathfrak{s}= s_1 * \cdots * s_\ell  \in \F^*(S)$, where each $s_i\in S$.
The \textbf{length} of $\mathfrak{s}$ is $| \mathfrak{s} | := \ell$.
(The empty word $\varepsilon_S$ is said to have length zero).

The elements $s_1,\dots,s_\ell$ are called the \textbf{factors} of $\mathfrak{s}$. 
A word $\mathfrak{t}\in \F^*(S)$ is said to be a \textbf{subword} of $\mathfrak{s}$ if there are $1\le i_1 < \dots < i_k \le \ell$ such that $\mathfrak{t} = s_{i_1}*\cdots*s_{i_k}$.
\end{defn}


\begin{defn} \label{def:factorization}
Let $H$ be a monoid.
The \textbf{factorization homomorphism} of $H$ is the unique homomorphism $\pi_H: \F^*(H) \to H$ satisfying $\pi_H(x) = x$ for all $x\in H$.

The \textbf{factorization monoid} of $H$ is the free monoid $\F^*(\A(H))$ generated by the atoms of $H$.
Its elements are referred to as \textit{factorizations}.

If $x\in H$ is a non-unit, then the \textbf{set of factorizations} of $x$ is
\[ \Z_H(x) := \{ \aa\in \F^*(\A(H)) : \pi_H(\aa) = x \} = \F^*(\A(H)) \cap \pi_H^{-1}(x) \]
The subscript ``$H$" may be omitted for brevity if the ambient monoid in which the factorization is being considered is clear from context.

For a non-empty word $\aa\in \Z_H(x)$, if we write $\aa = a_1*\cdots*a_k$, the atoms $a_i$ are said to be \textit{factors} of $x$.
\end{defn}

\begin{defn} \label{def:equivalence}
Let $H$ be a monoid, $x\in H$ be a non-unit, and $\aa,\bb \in \F^*(\A(H))$.
We will say that $\aa$ is \textbf{equivalent} to $\bb$ if, writing $\aa = a_1*\cdots * a_k$ and $\bb = b_1*\cdots * b_\ell$,
\begin{enumerate}
\item $k = \ell$.
\item The factors in $\bb$ are permuted associates of the factors of $\aa$; that is, there is a permutation $\sigma\in S_n$ (where $S_n$ is the symmetric group on $\llb 1,n \rrb$) such that $b_i \sim a_{\sigma(i)}$ for all $i\in \llb 1,k \rrb$.
\item $\aa$ and $\bb$ have the same product; i.e., $\pi_H(\aa) = \pi_H(\bb)$.
\end{enumerate}
\end{defn}

It is not difficult to check that the relation defined here is indeed an equivalence relation on $\F^*(\A(H))$.

\begin{defn} \label{def:set of lengths}
Let $H$ be a monoid and $x\in H\setminus H^\times$.
The \textbf{set of factorization classes} of $x$ is
\[ \mathsf{Z}_H(x) := \{ [\aa]: \aa\in \Z_H(x) \} = \Z_H(x)/\sim \]
and the \textbf{set of (factorization) lengths} of $x$ is
\[ \mathsf{L}_H(x) := \{ |\aa| : [\aa] \in \mathsf{Z}_H(x) \}. \]
Lastly, the \textbf{system of (sets of) lengths} of $H$ is $\mathcal{L}(H) := \{\mathsf{L}_H(x): x\in H\setminus H^\times\}$.
\end{defn}

\begin{defn} \label{def:factorization properties}
Let $H$ be a monoid.
Here we define some properties to measure the degree of uniqueness of factorization in $H$.
\begin{itemize}
\item $H$ has \textbf{unique factorization (UF)} if, for all $x\in H\setminus H^\times$, $|\mathsf{Z}_H(x)| = 1$ (we may also say $H$ is \textit{factorial}).
\item $H$ is \textbf{half factorial (HF)} if, for all $x\in H\setminus H^\times$, $|\mathsf{L}_H(x)| = 1$.
\item $H$ has \textbf{finite factorization (FF)} if, for all $x\in H\setminus H^\times$, $|\mathsf{Z}_H(x)| <\infty$.
\item $H$ has \textbf{bounded factorization (BF)} if, for all $x\in H\setminus H^\times$, $|\mathsf{L}_H(x)| < \infty$.
\item $H$ is \textbf{atomic} if, for all $x\in H\setminus H^\times$, $\mathsf{Z}_H(x) \neq \emptyset$.
\end{itemize}
\end{defn}

\begin{prop} \label{prop:factorization properties}
We have the following logical implications among the properties defined just above:
\[\begin{tikzcd}[arrows = Rightarrow]
 & \text{HF} \arrow{dr}&  &  \\
\text{UF} \arrow{ur}\arrow{dr}  &  & \text{BF} \arrow{r}& \text{atomic} \\
 & \text{FF} \arrow{ur} &  & 
\end{tikzcd}\]
\end{prop}



\section{Examples and Non-Examples} \label{sec:examples}
Here we will take a moment to examine the reversibility (or lack thereof) in the logical implications between the properties outlined in Proposition \ref{prop:factorization properties}.
We also remark on some well-studied classes of monoids which demonstrate some of those properties.

\begin{enumerate}[label={\rm (\roman{*})}]
\item $\ZZ\setminus \{0\}$ is a unique factorization monoid (this is the Fundamental Theorem of Arithmetic).
\item Let $\mathbb{P} \subseteq \NN$ be the set of primes, and let $M = \gen{\mathbb{P}\times \mathbb{P}}$ be the monoid generated by pairs of primes under multiplication.  
Then, for any pair $(m,n)\in M$, it is clear that any factorization of $(m,n)$ has length equal to the number of primes (counted with multiplicity) dividing $m$ or $n$.
However, this is not a UF monoid; we have, for instance, that $(2,2)(3,3)(2,3) = (12,18) = (2,3)(2,3)(3,2)$.
\item Most examples we will encounter from here onward will be FF, so it is perhaps more useful to see a non-example of an FF monoid.
Let $R = \mathbb{R} + x \mathbb{C}[x]$ be the ring of polynomials with complex coefficients and real constant term.
Then, for all nonzero $r\in \mathbb{R}$, we have
\[ x^2 = ((r+i)x)\left(\frac{1}{r+i}\,x\right). \]
Since $r+i\notin R$ for $r\neq0$, each $r+i$ is a non-unit of $R$, so we have found infinitely many factorizations of $x^2$.
However, any element of $R\setminus \{0\}$ has only finitely many factorization \textit{lengths} by a degree argument.
Thus the monoid $R\setminus \mathbb{Z}$ is BF but not FF.
\item Some of the richest factorization behavior is encountered in BF monoids.  
Here we mention some highly studied classes of BF monoids without going into too much detail, on the promise that we will discuss a new class of examples in heavy detail later.
\begin{itemize}
\item \textit{Numerical monoids}: proper subsets $H \subsetneqq \NN$ with finite complement which are closed under addition. \textcolor{red}{[CITE SOME PAPERS]}
\item \textit{Monoids of zero-sum sequences}: for a finite abelian group $G$, this monoid consists of formal words or ``sequences" in the elements of $G$ whose sums are equal to $0$.  
%This monoid can be written as $\F(G) \cap \sigma^{-1}(0)$, where $\F(G)$ is the free \textit{abelian} monoid in which words are unordered and $\sigma: \F(G) \to G$ is the unique homomorphism sending $g \mapsto g$ (compare with our product map $\pi_H: \F^*(H) \to H$).
The interest in these monoids can be traced back to the study of the class group of a Dedekind domain (usually a ring of integers of a number field).  \textcolor{red}{[CITE SOME PAPERS]}

\item \textit{Integer-valued polynomials}: let $D$ be a domain with field of fractions $K$; then $\operatorname{Int}(D) := \{f(x) \in K[x]: f(D) \subseteq D \}$ is the ring of integer-valued polynomials of $D$.
In addition to the rich theory developed around understanding the prime ideal structure of this ring, it is amenable to the study of factorization behavior, and exhibits some surprising behaviors.  
For example, any finite subset of $\NN_{\ge2}$ can be realized as the set of factorization lengths of some polynomial $f(x)\in \operatorname{Int}(D)$.
Additionally, one can pose similar questions regarding the ring $\operatorname{Int}^\text{R}(D)$ of integer-valued rational functions. \textcolor{red}{[CITE SOME PAPERS]}
\end{itemize}
\item Since we will usually be looking at atomic monoids, we offer a non-example here; consider the set $Q = \mathbb{Q}_{\ge 0}$ of non-negative rational numbers under addition.
$Q$ is reduced (its only unit is the identity, $0$) and we have, for any non-zero element $x\in Q$, that $x = \frac{x}{2} + \frac{x}{2}$.
This is a decomposition of $x$ into two non-zero (hence non-unit) elements, so $x$ is not an atom.
Thus we learn that $Q$ not only fails to have factorizations into atoms, but also to have atoms at all.
\end{enumerate}

\section{Monoid and Equimorphism Basics} \label{sec:monoids}
Here we will define and examine some properties that pertain to arbitrary monoids, as well as our main tool for transporting factorization information between arbitrary monoids: equimorphisms.



\begin{defn} \label{def:torsion}
	Let $H$ be a monoid.
	For any $x\in H$, let $\gen{x}_H := \{x^k: k\in \NN_{>0} \}$ denote the \textit{subsemigroup generated by $x$} in $H$.
	Then we define the \textbf{order} of $x$ in $H$ to be $\ord_H(x) := | \gen{x}_H|$ (we may drop the subscript $H$ when the monoid is clear from context).
	
	$H$ is said to be
	\begin{itemize}
		\item \textbf{torsion} if, for all $x\in H$, $\ord_H(x)<\infty$;
		\item \textbf{non-torsion} if there exists $x\in H$ with $\ord_H(x) = \infty$;
		\item \textbf{torsion-free} if, for all $x\in H$, $\ord_H(x) = \infty$.
	\end{itemize}
\end{defn}

Note that, in the case when $H$ is a group, $\gen{x}_H$ is the cyclic subgroup generated by $x$ in $H$, and $\ord_H(x)$ therefore corresponds to the familiar sense of ``order'' from group theory.


\begin{defn} \label{def:monoid properties}
	Let $H$ be a monoid and let $x,y,z\in H$.  
	\begin{itemize}
		\item \textbf{reduced} if $H^\times = \{1\}$ is trivial;
		\item \textbf{cancellative} if $xz = yz$ or $zx=zy$ implies that $x=y$;
		\item \textbf{unit cancellative} if $xy=x$ or $yx=x$ implies that $y\in H^\times$;
		\item \textbf{Dedekind-finite} if, $xy=1$ implies that $yx=1$ (i.e., one-sided inverses are actually two-sided).
	\end{itemize}
\end{defn}
The condition of Dedekind-finiteness will play a very important role in the remainder of this chapter. 
This is a very mild condition which one may almost expect of a ``typical'' monoid (we have cancellative $\Rightarrow$ unit-cancellative $\Rightarrow$ Dedekind-finite).
Example \ref{exa:no-dedekind-finiteness} includes a monoid which is \textit{not} Dedekind-finite.


\begin{prop}\label{prop:unit-adjust}
	Let $H$ be a monoid.
	The following hold:
	\begin{enumerate}[label={\rm (\roman{*})}]
		\item\label{it:prop:unit-adjust(0)} If $u, v \in H^\times$ then $uv \in H^\times$, and the converse holds whenever $H$ is Dedekind-finite.
		\item\label{it:prop:unit-adjust(0b)} If $\A(H)$ is non-empty or $H$ is commutative or unit-cancellative, then $H$ is Dedekind-finite.
		\item\label{it:prop:unit-adjust(i)} If $a\in \A(H)$ and $u, v \in H^\times$, then $uav\in \A(H)$.
		\item\label{it:prop:unit-adjust(ii)} If $x\in H\setminus H^\times$ and $u, v \in H^\times$, then $\mathsf{L}_H(uxv) = \mathsf{L}_H(x)$.
	\end{enumerate}
\end{prop}
%
\begin{proof}
	See \cite[parts (i), (ii), and (iv) of Lemma 2.2, and Proposition 2.30]{fan-tringali18}.
\end{proof}
%
%
\begin{defn} \label{def:divisor closed}
	Let $H$ be a monoid with a submonoid $M$.
	$M$ is \textbf{divisor closed} in $H$ if, whenever $y\in M$ and $x|_H y$, we have $x\in M$.
\end{defn}

Divisor closedness can help us gain a partial understanding of a monoid's factorization behavior by way of examining submonoids.
This strategy is made more precise by the following result, which is borrowed from \cite[Proposition 2.21]{fan-tringali18}.
\begin{prop} \label{prop:div closed factorization}
	Let $H$ be monoid with a divisor-closed submonoid $M$.
	Then
	\begin{enumerate}[label={\rm (\roman{*})}]
		\item $M^\times = H^\times$;
		\item $\A(M) = \A(H)\cap M$;
		\item For all $x\in M\setminus M^\times$, $\mathsf{Z}_M(x) = \mathsf{Z}_H(x)$;
		\item For all $x\in M\setminus M^\times$, $\mathsf{L}_M(x) = \mathsf{L}_H(x)$;
		\item $\mathcal{L}(M)\subseteq \mathcal{L}(H)$.
	\end{enumerate}
\end{prop}

\begin{proof}
	See the proof of \cite[Proposition 2.21]{fan-tringali18}.
\end{proof}

%
\begin{defn}\label{def:equimorphism}
	Let $H$ and $K$ be monoids, and $\varphi: H\to K$ a monoid homomorphism. We denote by $\varphi^*: \F^*(H)\to\F^*(K)$ the (unique) monoid homomorphism such that $\varphi^\ast(x) = \varphi(x)$ for every $x \in H$, and we call $\varphi$ an \emph{equimorphism} if the following hold:
	\begin{enumerate}[label={({\small{E}}\arabic{*})}]
		\item\label{def:equimorphism(E1)} $\varphi^{-1}(K^\times)\subseteq H^\times$;
		\item\label{def:equimorphism(E2)} $\varphi$ is \emph{atom-preserving}, meaning that $\varphi(\A(H)) \subseteq \A(K)$;
		\item\label{def:equimorphism(E3)} If $x\in H$ and $\mathfrak{b}\in \mathcal{Z}_K(\varphi(x))$ is a non-empty $\A(K)$-word, then $\varphi^*(\mathfrak{a}) \in [ \mathfrak{b} ]_{\C_K}$ for some $\mathfrak{a}\in \mathcal{Z}_H(x)$.
	\end{enumerate}
	Moreover, we say that $\varphi$ is \emph{essentially surjective} if $K = K^\times \varphi(H)K^\times$.
\end{defn}
%
\begin{prop}\label{prop:equimorphism}
	Let $H$ and $K$ be monoids and $\varphi:H\to K$ an equimorphism. The following hold:
	\begin{enumerate}[label={\rm (\roman{*})}]
		\item\label{it:prop:equimorphism(i)} $\mathsf{L}_H(x) = \mathsf{L}_K(\varphi(x))$ for all $x\in H\setminus H^\times$.
		\item\label{it:prop:equimorphism(ii)} If $\varphi$ is essentially surjective, then for all $y\in K\setminus K^\times$ there is $x\in H\setminus H^\times$ with $\mathsf{L}_K(y) = \mathsf{L}_H(x)$.
	\end{enumerate}
\end{prop}
%
\begin{proof}
	See \cite[Theorem 2.22(i)]{fan-tringali18} and \cite[Theorem 3.3(i)]{tringali18}.
\end{proof}
%

\section{Literature and Comparing Alternate Definitions} \label{sub:other-factorizations}
%
Our approach to factorization in possibly non-cancellative or non-commutative monoids is borrowed from \cite{fan-tringali18}, where one can read thoroughly about differences and similarities with 
the classical approach to factorization in commutative and cancellative monoids (and hence in integral domains) pursued by A. Geroldinger and F. Halter-Koch in \cite{geroldinger-hk06}, and with the much more recent approach to factorization in cancellative but possibly non-commutative monoids set forth by N.R. Baeth and D. Smertnig in \cite{baeth-smertnig15}; in particular, see \cite[Remarks 2.6 and 2.7]{fan-tringali18}.

This said, there are many previous entries in the literature that have treated (mainly algebraic) aspects of factorization theory in commutative (unital) rings with non-trivial zero divisors. Most notably, D.D. Anderson and collaborators have extensively studied factorizations in commutative rings corresponding to notions of ``associate'' and ``irreducible'' other than the ones adopted in the present paper, see e.g. \cite{AndMar85a,AndMar85b,AnVL96,AnVL97,AgAnVL01,chun-anderson11,chun-anderson-vl11}. Below we review these alternative definitions and contrast them with our approach. 

To start with, let $R$ be a commutative ring and denote by $R^\times$ the group of units of the multiplicative monoid of $R$. Given $x, y \in R$, we say in the parlance of \cite[Definition 2.1]{AnVL96} that 
\begin{itemize}
	\item $x$ is \emph{associate} to $y$ (in $R$), written $x \sim_R y$, if $xR = yR$;
	\item $x$ is \emph{strongly associate} to $y$, written $x \approx_R y$, if $x \in yR^\times$ (by Proposition \ref{prop:unit-adjust}\ref{it:prop:unit-adjust(0)}, this is equivalent to $x$ being associate (as per Definition \ref{defn:atom}) to $y$ in the multiplicative monoid of $R$);
	\item $x$ is \emph{very strongly associate} to $y$, written $x \cong_R y$, if $x \sim_R y$ and one of the following holds:
	\begin{enumerate}[label = {\rm (\roman{*})}] 
		\item $x = y = 0_R$ (where $0_R$ is the zero of $R$);
		\item $x \neq 0_R$ and if $x = yz$ for some $z \in R$ then $z \in R^\times$.
	\end{enumerate}
\end{itemize}
%
Accordingly, one has three notions of ``irreducible'', see \cite[Definition 2.4]{AnVL96}. To wit, an element $a \in R$ is 
\begin{itemize}
	\item \emph{irreducible} if $a \notin R^\times$ and $a=xy$ for some $x, y \in R$ implies that $a \sim_R x$ or $a \sim_R y$;
	\item \emph{strongly irreducible} if $a \notin R^\times$ and $a=xy$ for some $x, y \in R$ implies that $a \approx_R x$ or $a \approx_R y$;
	\item \emph{very strongly irreducible} if $a \notin R^\times$ and $a=xy$ for some $x, y \in R$ implies that $a \cong_R x$ or $a \cong_R y$.
\end{itemize}
It is obvious that very strongly irreducible elements of $R$ are strongly irreducible, and strongly irreducible elements are irreducible. In general, none of these implications can be reversed, see the paragraph after the proof of Theorem 2.12 in \cite{AnVL96}. However, we get by \cite[Theorem 2.2(3)]{AnVL96} that the three notions coincide when $R$ is \emph{pr\'esimplifiable} in the sense of \cite{Bou74a}, meaning that if $xy = x$ for some $x, y \in R$ then $x = 0_R$ or $y \in R^\times$ (e.g., this is the case when $R$ is an integral domain). Moreover, \cite[Theorem 2.5]{AnVL96} yields that a \emph{non-zero} element of $R$ is strongly irreducible if and only if it is an atom of the multiplicative monoid of $R$.

Putting it all together, we thus see that the ring $R$ is \emph{very strongly atomic} in the sense of \cite[Definition 3.1]{AnVL96} if and only if one of the following holds:
\begin{enumerate}[label={(\small{A}\arabic{*})}]
	\item\label{it:condition(A1)} $R$ has non-trivial zero divisors and the multiplicative monoid of $R$ is atomic (as per Section \ref{subsec:factorizations});
	\item\label{it:condition(A2)} $R$ is an integral domain and $R \setminus \{0_R\}$ is an atomic monoid under multiplication.
\end{enumerate}
Similarly, $R$ is a \emph{bounded factorization ring} in the sense of \cite[Definition 3.8]{AnVL96}, a \emph{half-factorial ring} in the sense of \cite[p. 87]{axtell-et-al03}, a \emph{finite factorization ring} in the sense of \cite[Definition 6.5]{AnVL96}, or a unique factorization ring in the sense of \cite[Definition 4.3]{AnVL96} if and only if one of conditions \ref{it:condition(A1)} and \ref{it:condition(A2)} in the above is satisfied with ``atomic'' replaced, respectively, by ``BF'', ``HF'', ``FF'', or ``factorial''.

As long as the scope is restricted to commutative rings, it is therefore possible to compare our approach to factorization with others based on irreducibles, strong irreducibles, or even alternative ``elementary factors'' (including the ones considered by C.R. Fletcher \cite{Fl69}, A. Bouvier \cite{Bou74a, Bou74b}, and S. Galovich \cite{Ga78}) by referring to \cite{AnVL96,chun-anderson-vl11}, where these comparisons are worked out in great detail. (Incidentally, it appears that Galovich is \emph{tacitly} assuming ``irreducibles'' in the sense of \cite{Ga78} to be non-units.) See also \cite[Theorem 3.4 and Corollary 3.5]{BaBuMi17} for a couple of results of a more arithmetic flavor concerning lengths of factorizations into irreducibles in commutative rings of the form $D/xD$, where $D$ is a principal ideal domain and $x$ is a non-zero, non-unit element of $D$. 
