\chapter{Introduction} \label{ch:intro}

Factorization theory pursues a full understanding of how complex objects decompose into their simplest constituent parts.  
Depending on the algebraic structure in question, the difficulty of gaining such an understanding can vary wildly. 
Some objects can be broken down in exactly one way, while others exhibit more exotic behavior and are able to be broken down into several qualitatively different combinations of simpler parts.
Among our tasks are to test the bounds of this behavior, and to completely classify the circumstances under which it can occur.
In the present work, we bring our attention to a fairly new class of algebraic objects -- the titular ``power monoids" --  which possess many characteristics that make their study difficult, hence interesting.

%%%%%%
%%%%%%
\section{A Brief History and Motivation}
%%%%%%
%%%%%%

The most elementary setting in which we study factorizations is the set $\ZZ$ of integers.
It is well known (as the Fundamental Theorem of Arithmetic) that every integer (other than $-1$, $0$, and $1$) factors uniquely as a product of prime integers.
For instance, $12$ can be written as $2\cdot 2\cdot 3$.
Of course, 12 can also be written as either of the products $2\cdot 3\cdot 2$ or $(-3)\cdot 2\cdot (-2)$, but we consider these factorizations to be fundamentally the same.  
This tells us that, in addition to identifying the prime factors involved, there should also be an equivalence of factorizations in play.
Making these ideas rigorous is one of the challenges of extending this familiar example to a more general theory.


There are many settings other than the integers in which it is reasonable to decompose elements into atoms. 
However, most will not share the familiar unique factorization of the integers.  
Historically, one of the greatest examples comes from the ring of integers of an algebraic number field; namely, $\ZZ[\sqrt{-5}]$.
Consider $6$ as an element of this ring: $6=2\cdot3$, but we also have $6 = (1+\sqrt{-5})(1-\sqrt{-5})$.
It is not a hard exercise to show that these two factorizations are not equivalent (meaning that $2$ and $3$ are not associate to $1\pm\sqrt{-5}$), so $6$ has more than one type of factorization into irreducibles.

%For an example, one doesn't need to migrate too far from the integers: consider the subset of integers which can be obtained as a sum of any number of copies of $4$, $9$, or $11$.
%The list of such integers begins: $\{0, 4, 8, 9, 11, 12, 13, 15, 16, 17, ... \}$.
%Here, it is necessary to determine which integers will appear in the list (one can show that every integer after $15$ will appear), but we will be more interested in the questions of how many ways integers can be written just using $4$, $9$, and $11$.
%As an example, $22$ can be written as $11+11$ or as $9+9+4$; so $22$ can be decomposed into two atoms or into three.
%This is an instance of an element whose factorizations aren't unique, but a fairly tame one relative to the sort of phenomena that can arise within the study of non-unique factorizations.

Much of the field has taken place in the setting of monoids which are not only commutative (satisfying $ab = ba$) but also cancellative, meaning that $ab = ac$ implies $b = c$.
It is true that monoids of ideals in commutative rings and monoids of modules naturally give rise to examples of non-cancellative settings in which it is reasonable to study factorization properties. 
Our goal here is to explore a relatively new class of monoids which are non-cancellative, exhibit many rich properties, and yet are rooted in a simple and natural combinatorial construction.


%%%%%%
%%%%%%
\section{Plan and Main Results}
%%%%%%
%%%%%%
%We will conclude this chapter by defining and recalling some preliminary notions which are necessary to move forward with our discussion, including the formal language we will use to encode the data of factorizations and some notions which capture the varying degrees of non-unique factorization.
In the remainder of this section, we will tell (some of) the story of factorization theory up to this point, by way of motivating the chapters to come.  
We will also set some notation and conventions to be used.

Chapter \ref{ch:fundamentals} begins by formulating the definitions necessary to have a detailed discussion of factorizations, including definitions which capture the various degrees to which a monoid may fail to have uniqueness of factorization.  
Following this is a discussion of some examples of these concepts.
From here, we outline a few more general monoid concepts needed for later, and close with a comparison of the present work to some previous entries in the literature and the alternate factorization theory notions therein.

Chapter \ref{ch:power monoids} will introduce our main object of study: the power monoid.  
In brief, for any monoid $H$, let $\P_\fin(H)$ be the collection of finite, nonempty subsets of $H$ with the operation of setwise multiplication given by $X\cdot Y = \{xy: x\in X, y\in Y\}$.  
This forms a monoid which can behave wildly.
However, the submonoid $\P_\fun(H)$ of subsets containing $1$ is more feasible for study, and yields some meaningful results which can be lifted back to the full monoid $\P_\fin(H)$ in some circumstances.
We will see in Section \ref{sec:atomicity} when it is reasonable to study factorizations in $\P_\fun(H)$.
As it turns out, this monoid is atomic exactly when $H$ has no nontrivial idempotents or elements of order $2$ (Theorem \ref{th:atomicity}).
Moreover, $\P_\fun(H)$ has bounded factorization lengths if and only if $H$ is torsion-free (Theorem \ref{thm:BF-torsion}).

When $H$ is not torsion-free, the usual tools for measuring non-uniqueness of factorization in $\P_\fun(H)$ become degenerate.
In Chapter \ref{ch:minimal factorizations}, we seek compensate for this by introducing the notion of \textit{minimal factorizations} in a general (non-cancellative) monoid.
We then classify the circumstances under which $\P_\fun(H)$ satisfies minimal versions of the usual factorization properties (Factoriality, Half-Factoriality, FF-ness, BF-ness) in Theorems \ref{BmF-char} and \ref{prop:HF-exp-3} and Corollary \ref{cor:when-reduced-pm-is-minimally-factorial}.
Finally, we move to the specific case when $H = \ZZ/n\ZZ$ is a finite cyclic group to exercise this new notion of minimal factorization and recover analogues of some results which are known for $\PN$.
Namely, each interval $\llb 2,k \rrb$ for $k\le n-1$ occurs as a set of lengths in $\P_\fon(\ZZ/n\ZZ)$ (Theorem \ref{th:interval-lengths}).

Chapter \ref{ch:partitions} focuses on the specific case of $\PN$.
In this setting we can leverage the linear ordering of $\NN$ to introduce the partition type of a given factorization.  
We consider the set of partition types of factorizations of a given element (by analogy with the more familiar set of lengths).
This is a new measure by which we can assess the degree to which an element fails to factor uniquely.  
In keeping with our previous findings, the intervals $\llb 0,n \rrb$ realizes all but $4$ possible partition types (Theorem \ref{thm:good types}), making them quantifiably the elements farthest from factoring uniquely.
Continuing to view $\PN$ through the lens of integer partitions, we also find another sharp bifurcation in factorization behavior between intervals and non-intervals: any non-interval $X$ satisfies $\max(\mathsf{L}_{\PN}(X) )\le \max(X)/2$.

Chapter \ref{ch:lattices} first establishes an intimate connection between the arithmetic of $\PN$ and $\P_\fon(\NN^d)$, for $d>1$; 
while it is clear that all phenomena encountered in $\PN$ can be found in $\P_\fon(\NN^d)$, the reverse is true as well (Theorem \ref{thm:passage-to-Nd}).  
This affords us the opportunity to use higher-dimensional geometric intuition to attack problems in $\PN$.
Ruminations in this vein bear the fruit of some new methods for understanding factorizations in $\P_\fon(\NN^d)$.
This line of thought also allows us to recover known results in $\PN$; namely, that, for any $n\ge 2$, $\{n\}$ and $\{2,n+1\}$ occur as sets of factorization lengths (Theorems \ref{thm:ind-atoms} and \ref{thm:2n-length-set}, respectively).
Finally, we push these methods further to obtain Theorem \ref{thm:int-point-construction}, which states that, for any $n\ge 2$ and $m\ge 1$, $\llb 2,m+2 \rrb \cup \{m+n+1\}$ occurs as a set of lengths.

\section{Notation and Conventions} \label{subsec:generalities}

\begin{itemize}
	\item $\NN = \{0,1,2,3,\dots\}$, $\ZZ = \{0,\pm 1, \pm 2,\dots\}$, and $\RR$ denote the sets of natural numbers, integers, and real numbers, respectively.
	Here we adopt the French convention that $0\in \NN$ so that $\NN$ is a monoid.
	\item For $k\in \NN$, we set the notation $\NN_{>k} := \{n\in \NN: n>k\}$ and $\NN_{\ge k} := \{n\in \NN: n\ge k\}$.
	\item In general, unless otherwise specified, lowercase letters ($a$, $b$, $x$, $y$, etc.) will usually refer to elements of a monoid; ordinary uppercase letters to sets and subsets ($A$, $B$, $X$, $Y$, etc.); script or calligraphic uppercase letters ($\A$, $\C$, $\F$, $\P$) to collections of subsets; math fraktur letters ($\aa$, $\bb$, etc.) to words in a free monoid over a given generating set (see the remainder of this section for more details on free monoids).
	\item Specifically, $i$, $j$, $k$, $\ell$, $m$, and $n$ will usually stand for non-negative integers. 
	$H$ will stand for a monoid, $G$ for a group, and $R$ for a ring.
	\item In general, we will refer to the operation of a not necessarily commutative monoid $(H,\cdot)$ as ``multiplication''.
	We will often adopt the practice of suppressing the symbol ``$\cdot$'' when no confusion is likely to arise; that is, we will write $xy$ instead of $x\cdot y$.
	\item The identity of a multiplicative monoid $(H,\cdot)$ will be called $1$; the identity of an additive monoid $(H,+)$ will be called $0$.
	\item $X\subseteq Y$ will mean that $X$ is a subset of $Y$; $X\subsetneqq Y$ will mean that $X\subseteq$ but $X\neq Y$; $K\le H$ will mean that $K$ is a submonoid of $H$.
	\item If $S$ is a set and $\E$ is an equivalence relation on $S$, the equivalence class of some $x\in S$ shall be denoted by $[x]_\E$.
	The subscript may be removed in situations where the implied equivalence is clear.
	\item For $a, b\in \RR\cup\{\infty\}$, $[ a,b ] = \{ n\in \ZZ : a \le n \le b \}$ shall denote the (integer) interval from $a$ to $b$.
	\item For a $k\in \NN$ and objects $x_i$ for $i\in [1,k]$, we will often write the family $X:=\{x_i: i\in [1,k]\}$ in ``long form'' as $x_1,\dots,x_k$.
	If $X$ is a subset of a monoid $(H,\cdot)$, we will write the ordered product of the elements of $X$ as $x_1\cdots x_k$.
	In particular, if $k$ happens to be $0$, this product is empty and we set the convention $x_1\cdots x_k = 1$ to avoid any confusion or ill-definedness.
\end{itemize}




%%%%%%
%%%%%%
