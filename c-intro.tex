\chapter{Introduction} \label{ch:intro}

Factorization theory pursues a full understanding of how complex objects decompose into their simplest constituent parts.  
Depending on the algebraic structure in question, the difficulty of gaining such an understanding can vary wildly. 
Some objects can be broken down in exactly one way, while others exhibit more exotic behavior and are able to be broken down into several qualitatively different combinations of simpler parts.
Among our tasks are to test the bounds of this behavior, and to completely classify the circumstances under which it can occur.
In the present work, we bring our attention to a fairly new class of algebraic objects -- the titular ``power monoids" --  which possess many characteristics that make their study difficult, hence interesting.

\section{Plan and Main Results}
%%%%%%
%%%%%%
%We will conclude this chapter by defining and recalling some preliminary notions which are necessary to move forward with our discussion, including the formal language we will use to encode the data of factorizations and some notions which capture the varying degrees of non-unique factorization.
In the remainder of this chapter, we will tell (some of) the story of factorization theory up to this point, by way of motivating the chapters to come.  
We will also set some notation and conventions to be used.

%Chapter \ref{ch:fundamentals} begins by formulating the definitions necessary to have a detailed discussion of factorizations, including definitions which capture the various degrees to which a monoid may fail to have uniqueness of factorization (the conditions UF, HF, FF, BF).
%Following this is a discussion of some examples of these concepts.
%From here, we outline a few more general monoid concepts needed for later, and close with a comparison of the present framework to the existing body of work on factorization theory and the alternative definitions therein.

Chapter \ref{ch:fundamentals} begins laying the foundation neessary to have a detailed discussion of factorizations.
Here we will give definitions and examples of properties which characterize the extend to which a monoid may fail to have uniqueness of factorization (the conditions UF, HF, FF, BF).  
The chapter ends with a comparison of the present framework to the existing body of literature, especially the work of D.D. Anderson et al.

%Chapter \ref{ch:power monoids} will introduce our main object of study: the power monoid.  
%In brief, for any monoid $H$, let $\P_\fin(H)$ be the collection of finite, nonempty subsets of $H$ with the operation of setwise multiplication given by $X\cdot Y = \{xy: x\in X, y\in Y\}$.  
%This forms a monoid which can behave wildly with respect to questions of factorization.
%However, the submonoid $\P_\fun(H)$ of subsets containing $1$ is more feasible for study, and yields some meaningful results which can be lifted back to the full monoid $\P_\fin(H)$ in some circumstances (for instance, when $H$ is a group).
%%We will see in Section \ref{sec:atomicity} when it is reasonable to study factorizations in $\P_\fun(H)$.
%As it turns out, $\P_\fun(H)$ is atomic exactly when $H$ has no nontrivial idempotents or elements of order $2$ (Theorem \ref{th:atomicity}).
%Moreover, $\P_\fun(H)$ has bounded factorization lengths if and only if $H$ is torsion-free (Theorem \ref{thm:BF-torsion}).

Chapter \ref{ch:power monoids} introduces the main object of study: the power monoid, $\P_\fin(H)$.
We discuss the reduction, possible in many cases of interest, to the reduced power monoid $\P_\fun(H)$.
We determine that $\P_\fun(H)$ is atomic exactly when $H$ has no nontrivial idempotent elements or elements of order $2$ (Theorem \ref{th:atomicity}) and that $\P_\fun(H)$ is BF exactly when $H$ is torsion free (Theorem \ref{thm:BF-torsion}).

%When $H$ is not torsion-free, the usual tools for measuring non-uniqueness of factorization in $\P_\fun(H)$ become degenerate.
%In Chapter \ref{ch:minimal factorizations}, we seek compensate for this by introducing the notion of \textit{minimal factorizations} in a general (non-cancellative) monoid.
%We then classify the circumstances under which $\P_\fun(H)$ satisfies minimal versions of the usual factorization properties (Factoriality, Half-Factoriality, FF-ness, BF-ness) in Theorems \ref{BmF-char} and \ref{prop:HF-exp-3} and Corollary \ref{cor:when-reduced-pm-is-minimally-factorial}.
%Finally, we move to the specific case when $H = \ZZ/n\ZZ$ is a finite cyclic group to exercise this new notion of minimal factorization and recover analogues of some results which are known for $\PN$.
%Namely, each interval $[ 2,k ]$ for $k\le n-1$ occurs as a set of lengths in $\P_\fon(\ZZ/n\ZZ)$ (Theorem \ref{th:interval-lengths}).

Chapter \ref{ch:minimal factorizations} returns to a discussion of factorization theory in general as we address the degeneracy of some of the usual notions in a non-cancellative setting.  
In response to this problem, we formulate the notion of a \textit{minimal factorization} and minimal versions of the usual factorization properties: UmF, HmF, FmF, BmF.
We then apply our new framework to power monoids to find that $\P_\fun(H)$ is BmF or FmF if and only if $\P_\fun(H)$ is atomic (Theorem \ref{BmF-char}) and that $\P_\fun(H)$ is HmF or UmF if and only if $H$ is a group of order dividing $3$ (Corollary \ref{cor:when-reduced-pm-is-minimally-factorial}).

In Chapter \ref{ch:partitions}, we begin to focus on the specific case of the reduced power monoid of $\NN$ with addition, which we refer to as $\PN$. 
In this setting we can leverage the linear ordering on $\NN$ to introduce the \textit{partition type} of a given factorization in $\PN$.
This is a new measure by which we may asses the degree to which an element fails to factor uniquely.
Indeed, we see that the intervals $[0,n]$ have factorizations of almost every possible partition type (Theorem \ref{thm:good types}).
This signifies a sharp dichotomy with the other elements of $\PN$, as any non-interval $X\in \PN$ satisfies $\mathsf{L}_\PN(X) \le \max(X)/2$ (Theorem \ref{thm:long factorization implies interval}).

%Chapter \ref{ch:partitions} focuses on the specific case of $\PN$.
%In this setting we can leverage the linear ordering of $\NN$ to introduce the partition type of a given factorization.  
%We consider the set of partition types of factorizations of a given element (by analogy with the more familiar set of lengths).
%This is a new measure by which we can assess the degree to which an element fails to factor uniquely.  
%In keeping with our previous findings, the intervals $[ 0,n ]$ realizes all but $4$ possible partition types (Theorem \ref{thm:good types}), making them quantifiably the elements farthest from factoring uniquely.
%Continuing to view $\PN$ through the lens of integer partitions, we also find another sharp bifurcation in factorization behavior between intervals and non-intervals: any non-interval $X$ satisfies $\max(\mathsf{L}_{\PN}(X) )\le \max(X)/2$.

Chapter 6 continues along the direction of quantifying the wild factorization behavior of intervals.
By constructing large families of atoms of $\PN$ and tying their growth rates to the growth of generalized Fibonacci numbers, we demonstrate that the number of factorizations of the interval $[0,n]$ grows exponentially with $n$.
In particular, for any $\e>0$, there is a constant $C$ such that, for sufficiently large $n$, $|\mathsf{Z}_\PN([0,n])| \ge C(\sqrt[4]{2}-\e)^n$ (Theorem \ref{thm:exponential intervals}).

%Chapter \ref{ch:lattices} first establishes an intimate connection between the arithmetic of $\PN$ and $\P_\fon(\NN^d)$, for $d>1$; 
%while it is clear that all phenomena encountered in $\PN$ can be found in $\P_\fon(\NN^d)$, the reverse is true as well (Theorem \ref{thm:passage-to-Nd}).  
%This affords us the opportunity to use higher-dimensional geometric intuition to attack problems in $\PN$ and leads us to develop some new methods for understanding factorizations in $\P_\fon(\NN^d)$.
%In turn, this line of thought allows us to recover known results in $\PN$; namely, that, for any $n\ge 2$, $\{n\}$ and $\{2,n+1\}$ occur as sets of factorization lengths (Theorems \ref{thm:ind-atoms} and \ref{thm:2n-length-set}, respectively).
%Finally, we push these methods further to obtain a new realization result for a family of length sets; namely, Theorem \ref{thm:int-point-construction}, which states that, for any $n\ge 2$ and $m\ge 1$, $[ 2,m+2 ] \cup \{m+n+1\}$ occurs as a set of lengths.

Chapter \ref{ch:lattices} first establishes that $\PN$ and $\P_\fon(\NN^d)$ have, in a sense suitable for our purposes, identical factorization behavior for all $d>1$.
This affords us the opportunity to use higher-dimensional geometric intuition to attack problems in $\PN$ and leads us to develop some new methods for understanding factorizations in $\P_\fon(\NN^d)$.
This line of thought allows us to recover some known results for realizations of length sets in $\PN$; namely, that $\{n\} \in\mathcal{L}(\PN)$ for all $n\ge 2$ (Theorem \ref{thm:ind-atoms} or  \cite[Proposition 4.9]{fan-tringali18}) and that $\{2,n+1\}\in \mathcal{L}(\PN)$ for all $n\ge 2$ (Theorem \ref{thm:2n-length-set} or \cite[Proposition 4.10]{fan-tringali18}).
Finally, we push the methods developed to realize a new family of sets as sets of lengths in $\PN$; we show that $[2,m+2]\cup\{m+n+1\}\in \mathcal{L}(\PN)$ for all $n\ge 2$ and $m\ge 1$ (Theorem \ref{thm:int-point-construction}).

%%%%%%
%%%%%%
\section{Brief History and Motivation}
%%%%%%
%%%%%%

The most elementary setting in which we study factorizations is the set $\ZZ$ of integers.
It is well known (as the Fundamental Theorem of Arithmetic) that every integer (other than $-1$, $0$, and $1$) factors uniquely as a product of prime integers.
For instance, $12$ can be written as $2\cdot 2\cdot 3$.
Of course, 12 can also be written as either of the products $2\cdot 3\cdot 2$ or $(-3)\cdot 2\cdot (-2)$, but we consider these factorizations to be fundamentally the same.  
This tells us that, in addition to identifying the prime factors involved, there should also be an equivalence of factorizations in play.
Making these ideas rigorous is one of the challenges of extending this familiar example to a more general theory.


There are many settings other than the integers in which it is reasonable to decompose elements into atoms. 
However, most will not share the familiar unique factorization of the integers.  
Historically, one of the greatest examples comes from the ring of integers of an algebraic number field; namely, $\ZZ[\sqrt{-5}]$.
Consider $6$ as an element of this ring: $6=2\cdot3$, but we also have $6 = (1+\sqrt{-5})(1-\sqrt{-5})$.
It is not a hard exercise to show that these two factorizations are not equivalent (meaning that $2$ and $3$ are not associate to $1\pm\sqrt{-5}$), so $6$ has more than one type of factorization into irreducibles.

%For an example, one doesn't need to migrate too far from the integers: consider the subset of integers which can be obtained as a sum of any number of copies of $4$, $9$, or $11$.
%The list of such integers begins: $\{0, 4, 8, 9, 11, 12, 13, 15, 16, 17, ... \}$.
%Here, it is necessary to determine which integers will appear in the list (one can show that every integer after $15$ will appear), but we will be more interested in the questions of how many ways integers can be written just using $4$, $9$, and $11$.
%As an example, $22$ can be written as $11+11$ or as $9+9+4$; so $22$ can be decomposed into two atoms or into three.
%This is an instance of an element whose factorizations aren't unique, but a fairly tame one relative to the sort of phenomena that can arise within the study of non-unique factorizations.

Much of the field has taken place in the setting of monoids which are not only commutative (satisfying $ab = ba$) but also cancellative, meaning that $ab = ac$ implies $b = c$.
It is true that monoids of ideals in commutative rings and monoids of modules naturally give rise to examples of non-cancellative settings in which it is reasonable to study factorization properties. 
Our goal here is to explore a relatively new class of monoids which are non-cancellative, exhibit many rich properties, and yet are rooted in a simple and natural combinatorial construction.


%%%%%%
%%%%%%


\section{Notation and Conventions} \label{subsec:generalities}

\begin{itemize}
	\item $\NN = \{0,1,2,3,\dots\}$, $\ZZ = \{0,\pm 1, \pm 2,\dots\}$, and $\RR$ denote the sets of natural numbers, integers, and real numbers, respectively.
	Here we adopt the French convention that $0\in \NN$ so that $\NN$ is a monoid.
	
	\item In general, unless otherwise specified, lowercase letters ($a$, $b$, $x$, $y$, etc.) will usually refer to elements of a monoid; 
	ordinary uppercase letters to sets and subsets ($A$, $B$, $X$, $Y$, etc.); 
	script or calligraphic uppercase letters ($\A$, $\C$, $\F$, $\P$) to collections of subsets; 
	math fraktur letters ($\aa$, $\bb$, etc.) to words in a free monoid over a given generating set 
	(see Section \ref{subsec:factorizations} for more details on free monoids).
	Specifically, $i$, $j$, $k$, $\ell$, $m$, and $n$ will usually stand for non-negative integers. 
	$H$ will stand for a monoid, $G$ for a group, and $R$ for a ring.
	
	\item $X\subseteq Y$ will mean that $X$ is a subset of $Y$; $X\subsetneqq Y$ will mean that $X\subseteq Y$ but $X\neq Y$; $K\le H$ will mean that $K$ is a submonoid of $H$.
	
	\item For a subset $S\subseteq \RR$ and $k\in \NN$, we set the notation $S_{>k} := \{n\in S: n>k\}$ and $S_{\ge k} := \{n\in S: n\ge k\}$.
	
	\item For $a, b\in \RR\cup\{\infty\}$, $[ a,b ] = \{ n\in \ZZ : a \le n \le b \}$ shall denote the (integer) interval from $a$ to $b$.
	
	\item For a real number $x$, $\lfloor x \rfloor := \max\{n\in \ZZ: n \le x\}$ (read \textit{floor of} $x$) is the greatest integer less than $x$.
	Similarly, $\lceil x \rceil := \min\{n\in \ZZ: n\ge x\}$ (read \textit{ceiling of} $x$) is the least integer greater than $x$.
	
	\item If $S$ is a set and $\E$ is an equivalence relation on $S$, the equivalence class of some $x\in S$ shall be denoted by $[x]_\E$.
	The subscript may be removed in situations where the implied equivalence is clear.
	
	\item Fix an integer $m>1$.  When $m$ is understood from context, $\bar{a} \in \ZZ/m\ZZ$ will denote the residue class of $a$ modulo $m$.
	We will write $a\equiv b \mod m$ if $\bar{a} = \bar{b}$.
	
	\item On occasion, we will wish to lift residue classes to elements of $\NN$.  
	If $x\in \ZZ/m\ZZ$ and $m$ is well understood from context, we set $\hat{x} := \min(x\cap \NN) \in [0,m-1]$.
	By a similar token, for a subset $X\subseteq \ZZ/m\ZZ$, we set $\hat{X} := \{\hat{x}: x\in X\}$.
	
	\item In general, we will refer to the operation of a not necessarily commutative monoid $(H,\cdot)$ as ``multiplication''.
	We will often adopt the practice of suppressing the symbol ``$\cdot$'' when no confusion is likely to arise; that is, we will write $xy$ instead of $x\cdot y$.
	
	\item The identity of a multiplicative monoid $(H,\cdot)$ will be called $1$; the identity of an additive monoid $(H,+)$ will be called $0$.
	
	\item For a $k\in \NN$ and elements $x_i$ of some set for $i\in [1,k]$, we will often write the family $X:=\{x_i: i\in [1,k]\}$ in ``long form'' as $x_1,\dots,x_k$.
	On the other hand, we may refer to its elements simply ``the $x_i$'' or $\{x_i\}_i$ for brevity.
	If $X$ is a subset of a monoid $(H,\cdot)$, we will write the ordered product of the elements of $X$ as $x_1\cdots x_k$.
	In particular, if $k$ happens to be $0$, this product is empty and we set the convention $x_1\cdots x_k = 1$ to avoid any confusion or ill-definedness.
\end{itemize}




%%%%%%
%%%%%%
