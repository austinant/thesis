\chapter{Power Monoids and (Minimal) Factorization Properties} \label{ch:power monoids}
%%%%%%
%%%%%%



We begin by defining our central object of study: the power monoid.
These objects were first introduced and studied by Y. Fan and S. Tringali in \cite{fan-tringali18}.

\begin{defn}
Let $H$ be a monoid; for nonempty $X,Y\subset H$, we will define the operation of setwise multiplication by
\[X \cdot Y = \{xy : x\in X,\, y\in Y \}. \]
This operation endows several collections of subsets of $H$ with a monoid structure.  
Namely, we have the \textit{Power Monoid} of $H$:
\[ \P_\fin(H) = \{ X \subseteq H: X\neq \emptyset,\, |X| < \infty \}\]
the \textit{Restricted Power Monoid} of $H$:
\[ \P_\funt(H) = \{ X \subseteq H: X \cap H^\times \neq \emptyset,\,|X| < \infty \} \]
and the \textit{Reduced Power Monoid} of $H$:
\[ \P_\fun(H) = \{ X \subseteq H: 1\in X, \,  |X| < \infty \}. \]
\end{defn}

\begin{rk} Let $H$ be a monoid.
\begin{enumerate}[label={\rm (\roman{*})}]
\item $\P_\fin(H) \supseteq \P_\funt(H) \supseteq \P_\fun(H)$.
\item The identity of each of these monoids is $\{1_H\}$.
Moreover, $\P_\fun(H)$ is indeed a reduced monoid; i.e., its only unit is $\{1_H\}$.
\item Unless $H$ is trivial, $\P_\fin(H)$ is non-cancellative.
\end{enumerate}
\end{rk}

%%%%%%
%%%%%%
\section{Conditions for Atomicity and Bounded Factorization Lengths} \label{sec:atomicity}
%%%%%%
%%%%%%

Here we embark on the study of the (arithmetic and algebraic) structure of power monoids.
We begin with some elementary but helpful observations we will often use without comment.
%
\begin{prop}\label{prop:unit-adjust}
	Let $H$ be a monoid.
	The following hold:
	\begin{enumerate}[label={\rm (\roman{*})}]
		\item\label{it:prop:unit-adjust(0)} If $u, v \in H^\times$ then $uv \in H^\times$, and the converse is also true if $H=H^\times$ or $\mathcal{A}(H)=\emptyset$.
		\item\label{it:prop:unit-adjust(i)} If $a\in \mathcal{A}(H)$ and $u, v \in H^\times$, then $uav\in \mathcal{A}(H)$.
		\item\label{it:prop:unit-adjust(ii)} If $x\in H\setminus H^\times$ and $u, v \in H^\times$, then $\mathsf{L}_H(uxv) = \mathsf{L}_H(x)$.
	\end{enumerate}
\end{prop}
%
\begin{proof}
	See parts (i), (ii), and (iv) of \cite[Lemma 2.2]{fan-tringali18}.
\end{proof}

\begin{prop}\label{prop:pm-arith}
	Let $H$ be a monoid. The following hold:
	\begin{enumerate}[label={\rm (\roman{*})}]
		\item\label{it:prop:pm-arith(i)} If $X_1,\ldots,X_n\in \P_{\fin,1}(H)$, then $X_1 \cup \cdots \cup X_n \subseteq X_1 \cdots X_n$.
		%
		\item\label{it:prop:pm-arith(ii)} If $u,v \in H^\times$ and $X_1,\ldots,X_n\in \P_{\fin,\times}(H)$, then $|uX_1 \cdots X_nv| = |X_1 \cdots X_n| \ge \max_{1 \le i \le n} |X_i|$.
		%
		\item\label{it:prop:pm-arith(iii)} If $K$ is a submonoid of $H$, then $\P_\fun(K)$ is a divisor-closed submonoid of $\P_\fun(H)$.
		%
		\item \label{it:prop:pm-arith(iv)} $\P_\fun(H)$ is a reduced monoid and $\P_\fin(H)^\times=\P_\funt(H)^\times = \bigl\{ \{u\}: u\in H^\times\bigr\}$.
		\item \label{it:prop:pm-arith(v)} $\mathcal{A}(\mathcal P_{\funt}(H)) \subseteq H^\times \mathcal{A}(\mathcal P_{\fun}(H)) H^\times$.
	\end{enumerate}
\end{prop}
%
\begin{proof}
	\ref{it:prop:pm-arith(i)} is trivial, upon considering that $(X \cdot 1_H) \cup (1_H \cdot Y) \subseteq XY$ for all $X, Y \in \mathcal P_\fun(H)$; \ref{it:prop:pm-arith(ii)} is a direct consequence of \ref{it:prop:pm-arith(i)} and the fact that the function $X \to H: x \mapsto uxv$ is injective for all $u, v \in H^\times$ and $X \subseteq H$; and \ref{it:prop:pm-arith(iii)} and \ref{it:prop:pm-arith(iv)} are immediate from \ref{it:prop:pm-arith(i)} and \ref{it:prop:pm-arith(ii)}.
	
	As for \ref{it:prop:pm-arith(v)}, let $A\in \mathcal{A}(\P_\funt(H))$.
	Because $A$ contains a unit of $H$, there is $u\in H^\times$ such that $1_H \in uA$.
	Then $uA$ is an element of $\P_\fun(H)$, and by Proposition \ref{prop:unit-adjust}\ref{it:prop:unit-adjust(i)} it is also an atom of $\P_\funt(H)$.
	Thus, if $X,Y\in \P_\fun(H) \subseteq \P_\funt(H)$ and $uA = XY$, then $X$ or $Y$ is the identity of $\P_\fun(H)$.
	This means that $uA$ is an atom of $\P_\fun(H)$, and hence $A=u^{-1}(uA)\in H^\times \mathcal{A}(\P_\fun(H))$, as wished.
\end{proof}
%
%
Our ultimate goal is, for an arbitrary monoid $H$, to investigate factorizations in $\P_\fin(H)$. However, this is a difficult task in general, due to a variety of ``pathological situations'' that might be hard to classify in a satisfactory way, see, for instance, \cite[Remark 3.3(ii)]{fan-tringali18}.

In practice, it is more convenient to start with $\P_\fun(H)$ and then lift arithmetic results from $\P_\fun(H)$ to $\P_\funt(H)$, a point of view which is corroborated by the simple consideration that $\P_\fin(H) = \P_\funt(H)$ whenever $H$ is a group (i.e., in the case of greatest interest in Arithmetic Combinatorics).

In turn, we will see that studying the arithmetic of $\P_\funt(H)$ is tantamount to studying that of $\P_\fun(H)$, in a sense to be made precise presently.
To do so in as all-encompassing a way as possible, we recall from \cite[Definition 3.2]{tringali18} a notion which formally packages the idea that, under suitable conditions, arithmetic may be transferred from one monoid to another.
%
\begin{defn}\label{def:equimorphism}
	Let $H$ and $K$ be monoids, and $\varphi: H\to K$ a monoid homomorphism. We denote by $\varphi^*: \mathcal{F}^*(H)\to\mathcal{F}^*(K)$ the (unique) monoid homomorphism such that $\varphi^\ast(x) = \varphi(x)$ for every $x \in H$, and we call $\varphi$ an \textit{equimorphism} if the following hold:
	\begin{enumerate}[label={({\small{E}}\arabic{*})}]
		\item\label{def:equimorphism(E1)} $\varphi^{-1}(K^\times)\subseteq H^\times$;
		\item\label{def:equimorphism(E2)} $\varphi$ is \textit{atom-preserving}, meaning that $\varphi(\mathcal{A}(H)) \subseteq \mathcal{A}(K)$;
		\item\label{def:equimorphism(E3)} If $x\in H$ and $\mathfrak{b}\in \mathcal{Z}_K(\varphi(x))$ is a non-empty $\mathcal{A}(K)$-word, then $\varphi^*(\mathfrak{a}) \in \llb \mathfrak{b} \rrb_{\mathcal{C}_K}$ for some $\mathfrak{a}\in \mathcal{Z}_H(x)$.
	\end{enumerate}
	Moreover, we say that $\varphi$ is \textit{essentially surjective} if $K = K^\times \varphi(H)K^\times$.
	% (here as usual, we identify $H$ with a subset of $\mathcal{F}^\ast(H)$).
\end{defn}
%
\begin{prop}\label{prop:equimorphism}
	Let $H$ and $K$ be monoids and $\varphi:H\to K$ an equimorphism. The following hold:
	\begin{enumerate}[label={\rm (\roman{*})}]
		\item\label{it:prop:equimorphism(i)} $\mathsf{L}_H(x) = \mathsf{L}_K(\varphi(x))$ for all $x\in H\setminus H^\times$.
		\item\label{it:prop:equimorphism(ii)} If $\varphi$ is essentially surjective, then for all $y\in K\setminus K^\times$ there is $x\in H\setminus H^\times$ with $\mathsf{L}_K(y) = \mathsf{L}_H(x)$.
	\end{enumerate}
\end{prop}
%
\begin{proof}
	See \cite[Theorem 2.22(i)]{fan-tringali18} and \cite[Theorem 3.3(i)]{tringali18}.
\end{proof}
%
\begin{prop}\label{prop:funt&fun-have-the-same-system-of-lengths}
	Let $H$ be a Dedekind-finite monoid. The following hold:
	%
	\begin{enumerate}[label={\rm (\roman{*})}]
		\item\label{it:prop:funt&fun-have-the-same-system-of-lengths(i)} The natural embedding $\jmath: \P_\fun(H)\hookrightarrow \P_\funt(H)$ is an essentially surjective equimorphism.
		%
		\item\label{it:prop:funt&fun-have-the-same-system-of-lengths(ii)} $\mathcal{A}(\mathcal P_{\funt}(H)) = H^\times \mathcal{A}(\mathcal P_{\fun}(H)) H^\times$.
		%
		\item\label{it:prop:funt&fun-have-the-same-system-of-lengths(iii)} $\mathsf{L}_{\mathcal P_{\fun}(H)}(X) = \mathsf{L}_{\mathcal P_{\funt}(H)}(X)$ for every $X \in \mathcal P_{\fun}(H)$.
		%
		\item\label{it:prop:funt&fun-have-the-same-system-of-lengths(iv)} $\mathcal{L}(\mathcal P_{\funt}(H)) = \mathcal{L}(\mathcal P_{\fun}(H))$.
	\end{enumerate}
\end{prop}
%
\begin{proof}
	%	
	In view of Proposition \ref{prop:equimorphism}, parts \ref{it:prop:funt&fun-have-the-same-system-of-lengths(iii)} and \ref{it:prop:funt&fun-have-the-same-system-of-lengths(iv)} are immediate from \ref{it:prop:funt&fun-have-the-same-system-of-lengths(i)}.
	Moreover, the inclusion from left to right in \ref{it:prop:funt&fun-have-the-same-system-of-lengths(ii)} is precisely the content of Proposition \ref{prop:pm-arith}\ref{it:prop:pm-arith(v)}, and the other inclusion will follow from \ref{it:prop:funt&fun-have-the-same-system-of-lengths(i)} and Propositions \ref{prop:unit-adjust}\ref{it:prop:unit-adjust(i)} and \ref{prop:pm-arith}\ref{it:prop:pm-arith(iv)}.
	Therefore, we focus on \ref{it:prop:funt&fun-have-the-same-system-of-lengths(i)} for the remainder of the proof.
	
	\ref{it:prop:funt&fun-have-the-same-system-of-lengths(i)} By Proposition \ref{prop:pm-arith}\ref{it:prop:pm-arith(iv)}, $\jmath$ satisfies \ref{def:equimorphism(E1)}.
	Moreover, $\jmath$ is essentially surjective, as any $X\in \mathcal P_\funt(H)$ contains a unit $u \in H^\times$, so $u^{-1}X\in \mathcal P_\fun(H)$ and $X = u(u^{-1}X)$ is associate to an element of $\mathcal P_\fun(H)$. 
	
	To prove \ref{def:equimorphism(E2)}, let $A\in\mathcal{A}(\mathcal P_\fun(H))$.
	We aim to show that $A$ is an atom of $\mathcal P_\funt(H)$. For, suppose that $A = XY$ for some $X,Y\in \mathcal P_\funt(H)$. Then there are $x \in X$ and $y \in Y$ with
	$xy=1_H$; and using that $H$ is Dedekind-finite, we get from \cite[Lemma 2.30]{fan-tringali18} that $x,y\in H^\times$.
	It follows that 
	\[
	A = XY = (Xx^{-1})(xY)
	\quad\text{and}\quad
	Xx^{-1}, xY\in \mathcal P_\fun(H). 
	\]
	But then
	$Xx^{-1} = \{1_H\}$ or $xY = \{1_H\}$, since $\mathcal P_\fun(H)$ is a reduced monoid and $A$ is an atom of $\mathcal P_\fun(H)$.
	So, $X$ or $Y$ is a $1$-element subset of $H^\times$, and hence $A \in \mathcal{A}(\mathcal P_\funt(H))$. 
	
	
	It remains to show that $\jmath$ satisfies \ref{def:equimorphism(E3)}. For, pick $X \in \mathcal P_\fun(H)$. If $X = \{1_H\}$, the conclusion holds vacuously. Otherwise, let $\mathfrak{b} := B_1*\cdots*B_n \in \mathcal Z_{\mathcal P_\fun(H)}(X)$. Then there are $u_1\in B_1,\ldots, u_n\in B_n$ such that $1_H = u_1\cdots u_n$; and as in the proof of \ref{def:equimorphism(E2)}, it must be that $u_1,\ldots, u_n\in H^\times$.
	Accordingly, we take, for every $i \in \llb 1, n \rrb$, $A_i := u_0 \cdots u_{i-1} B_i u_i^{-1} \cdots u_1^{-1}$, where $u_0 := 1_H$.
	Then 
	\[
	A_1 \cdots A_n = X
	\quad\text{and}\quad
	1_H \in A_1, \ldots, 1_H \in A_n;
	\]
	and by Propositions \ref{prop:unit-adjust}\ref{it:prop:unit-adjust(i)} and \ref{prop:pm-arith}\ref{it:prop:pm-arith(v)}, $A_1, \ldots, A_n$ are atoms of $\mathcal P_{\fun}(H)$. This shows that $\mathfrak{a}:= A_1*\cdots * A_n \in \mathcal{Z}_{\mathcal P_\fun(H)}(X)$; and since $A_i \simeq_{\mathcal P_\funt(H)} B_i$ for each $i \in \llb 1, n \rrb$ (by construction), we conclude that $\mathfrak{a}$ is $\mathcal{C}_{\mathcal P_{\funt}(H)}$-congruent to $\mathfrak{b}$, as wished.
	%	
\end{proof} 
%
The next example shows that Dedekind-finiteness is, to some extent, necessary for Proposition \ref{prop:funt&fun-have-the-same-system-of-lengths}\ref{it:prop:funt&fun-have-the-same-system-of-lengths(ii)}, and hence for the subsequent conclusions. 
%
\begin{eg}\label{exa:no-dedekind-finiteness}
	Let $\mathcal{B}$ be the set of all binary sequences $\mathfrak{s}: \NN^+ \to \{0,1\}$, and let $H$ denote the monoid of all functions $\mathcal{B} \to \mathcal{B}$ under composition: We will write $\mathcal{B}$ multiplicatively; so, if $f, g \in \mathcal{B}$ then $fg$ is the map $\mathcal{B} \to \mathcal{B}: \mathfrak{s} \mapsto f(g(\mathfrak s))$.
	Further, let $n\ge 5$ and consider the functions 
	\begin{align*}
	L: \mathcal{B} \to\mathcal{B}: (a_1,a_2,\dots) &\mapsto (a_2,a_3,\dots)  &\,\textrm{(left shift);}\\
	R: \mathcal{B} \to\mathcal{B}: (a_1,a_2,\dots) &\mapsto (0,a_1,a_2,\dots) &\,\textrm{(right shift);}\\
	P: \mathcal{B} \to\mathcal{B}: (a_1,a_2,\dots) &\mapsto (a_{n},a_1,\dots, a_{n-1}, a_{n+2}, a_{n+3},\dots,) &\,\textrm{(cycle the first $n$ terms).}
	\end{align*}
	%
	In particular, $P \in H^\times$. Also, $LR = \operatorname{id}_\mathcal{B}$ but $RL \neq \operatorname{id}_\mathcal{B}$; whence $H$ is not Dedekind-finite, and neither $R$ nor $L$ is invertible. With this in mind, we will prove that $A:= \{L,P\} \cdot \{R,P\} = \{\operatorname{id}_\mathcal{B}, LP, PR, P^2 \}$ is an atom of $\P_\fun(H)$, although it is not, by construction, an atom of $\P_\funt(H)$.
	
	Indeed,
	assume $A = XY$ for some $X,Y\in \P_\fun(H)$.
	Then $X,Y\subseteq A$, and it is clear that $P^2 \ne PRLP$, or else $RL = {\rm id}_\mathcal{B}$ (a contradiction). Similarly, $PRPR \ne P^2 \ne LPLP$; otherwise, $P = RPR$ and hence $R$ is invertible, or $P = LPL$ and $L$ is invertible (again a contradiction). Lastly, we see that $P^2 \ne LP^2 R$ (by applying both functions to the constant sequence $1, 1, \ldots$).
	
	It follows that $P^2$ must belong to $X$ or $Y$, but not to both (which is the reason for choosing $n\ge 5$).
	%or $Y$ (since $P^2$ must be a product of units), but not to both .
	Accordingly, let $P^2 \in X \setminus Y$ (the other case is analogous).
	Then $Y = \{\operatorname{id}_\mathcal{B}\}$, since one can easily check that that $P^2LP, P^3R \notin A$, by noting that the action of $P^2LP$ and $P^3R$ differ from that of $A$ on the sequences $(1,1,\ldots)$ and $(1,0,1,1,\ldots)$. This makes $A$ an atom of $\P_\fun(H)$.
\end{eg}
%
We get from Proposition \ref{prop:funt&fun-have-the-same-system-of-lengths} that studying factorization properties of $\P_\fun(H)$ is sufficient for studying corresponding properties of $\P_\funt(H)$, at least in the case when $H$ is Dedekind-finite.
Thus, as a starting point in the investigation of the arithmetic of $\P_\fun(H)$, one might wish to give a comprehensive description of
the atoms of $\P_\fun(H)$.
This is however an overwhelming task even in specific cases (e.g., when $H$ is the additive group of the integers), let alone the general case. Nevertheless, we can obtain basic information about $\mathcal{A}(\P_\fun(H))$ in full generality.
%
%
\begin{lemma}\label{lem:2-elt-atoms}
	Let $H$ be a monoid and $x \in H \setminus \{1_H\}$.
	The following hold:
	\begin{enumerate}[label={\rm (\roman{*})}]
		%
		\item\label{it:lem:2-elt-atoms(i)} The set $\{1_H, x\}$ is an atom of $\P_\fun(H)$ if and only if $1_H \ne x^2 \ne x$.
		\item\label{it:lem:2-elt-atoms(ii)} If $x^2=1_H$ or $x^2=x$, then $\{1_H,x\}$ does not factor into a product of atoms neither in $\P_{\fin,1}(H)$ nor in $\P_\funt(H)$.
	\end{enumerate}
\end{lemma}
%
\begin{proof}
	\ref{it:lem:2-elt-atoms(i)} If $x^2 = 1_H$ or $x^2 = x$, then it is clear that $\{1_H,x\} = \{1_H,x\}^2$, and therefore $\{1_H,x\}$ is not an atom of $\mathcal P_\fun(H)$. As for the converse, assume that $\{1_H,x\} = YZ$ for some non-units $Y,Z \in \mathcal P_\fun(H)$. Then we get from Proposition \ref{prop:pm-arith} that $Y$ and $Z$ are $2$-element sets, namely, $Y = \{1_H, y\}$ and $Z = \{1_H, z\}$ with $y,z \in H \setminus \{1_H\}$. Hence $\{1_H,x\} = YZ = \{1_H,y,z,yz\}$, and immediately this implies $x=y=z$. Therefore, $\{1_H,x\} = \{1_H,x,x^2\}$, which is only possible if $x^2 = 1_H$ or $x^2 = x$.
	
	\ref{it:lem:2-elt-atoms(ii)} Suppose that $x^2 = 1_H$ or $x^2 = x$. Then the calculation above shows that $\{1_H,x\} = \{1_H,x\}^2$ and there is no other decomposition of $\{1_H,x\}$ into a product of non-unit elements of $\mathcal P_{\fin,1}(H)$. So, $\{1_H,x\}$ is a non-trivial idempotent (hence, a non-unit) and has no factorization into atoms of $\mathcal P_{\fin,1}(H)$.
	
	It remains to prove the analogous statement for $\mathcal P_{\fin,\times}(H)$.
	For, assume to the contrary that $\{1_H,x\}$ factors into a product of $n$ atoms of $\mathcal P_{\fin,\times}(H)$ for some $n \in \NN^+$. Then $n \ge 2$, since $\{1_H, x\}$ is a non-trivial idempotent (and hence not an atom itself). Consequently, we can write $\{1_H,x\} = YZ$, where $Y$ is an atom and $Z$ a non-unit of $\mathcal P_{\fin,\times}(H)$. In particular, we get from parts \ref{it:prop:pm-arith(i)}, \ref{it:prop:pm-arith(ii)}, and \ref{it:prop:pm-arith(iv)} of Proposition \ref{prop:pm-arith} that both $Y$ and $Z$ are $2$-element sets, say, $Y = \{u, y\}$ and $Z = \{v, z\}$. It is then immediate that there are only two possibilities: $1_H$ is the product of two units from $Y$ and $Z$, or the product of two non-units from $Y$ and $Z$.
	Without loss of generality, we are thus reduced to considering the following cases.
	\vskip 0.1cm
	\textsc{Case 1:} $uv=1_H$. Then $uz \ne 1_H$ (or else $z = u^{-1} = v$, contradicting the fact that $Z$ is a $2$-element set). So $uz=x$, and similarly $yv=x$. Then $y = xu = uzu$ and $z = xv = vyv$, and therefore
	\[
	\{u,y\} = \{u,uzu\} = \{1_H,uz\}\cdot \{u\} = \{1_H,x\}\cdot\{u\} = \{u,y\}\cdot \{vu,zu\}
	\]
	However, this shows that $\{u,y\}$ is not an atom of $\P_\funt(H)$, in contrast with our assumptions.
	\vskip 0.1cm
	\textsc{Case 2:}
	$yz = 1_H$ and $y,z\in H\setminus H^\times$. Then $u,v\in H^\times$, by the fact that $\{u, y\}, \{v, z\} \in \mathcal P_{\fin,\times}(H)$; and we must have $uz=x$, for $uz=1_H$ would yield $z = u^{-1}\in H^\times$.
	In particular, $x = uz$ is not a unit in $H$, so $uv=1_H$ and we are back to the previous case.
\end{proof}
%
We have just seen that, to even {\it hope} for $\P_\fun(H)$ to be atomic, we must have that the ``bottom layer'' of $2$-element subsets of $H$ consists only of atoms, and it will turn out that such a condition is also sufficient.
Before proving this, it seems appropriate to point out some structural implications of the fact that every non-identity element of $H$ is neither an idempotent nor a square root of $1_H$.
%
\begin{lemma}\label{lem:no-non-id-elts-of-small-order-implies-structure}
	Let $H$ be a monoid such that $1_H \ne x^2 \ne x$ for all $x\in H\setminus\{1_H\}$. The following hold:
	%
	\begin{enumerate}[label={\rm (\roman{*})}]
		\item\label{it:lem:no-non-id-elts-of-small-order-implies-structure(i)}
		$H$ is Dedekind-finite.
		\item\label{it:lem:no-non-id-elts-of-small-order-implies-structure(ii)}
		If $x \in H$ and $\gen{x}_H$ is finite, then $x \in H^\times$ and $\gen{x}_H$ is a cyclic group of order $\ge 3$.
	\end{enumerate}
\end{lemma}
%
\begin{proof}
	\ref{it:lem:no-non-id-elts-of-small-order-implies-structure(i)}
	Let $y,z\in H$ such that $yz = 1_H$. Then $(zy)^2 = z(yz)y = zy$, and since $H$ has no non-trivial i\-dem\-po\-tents, we conclude that $zy=1_H$. Consequently, $H$ is Dedekind-finite.
	
	\ref{it:lem:no-non-id-elts-of-small-order-implies-structure(ii)}
	This is an obvious consequence of \cite[Ch. V, Exercise 4, p. 68]{whitelaw}, according to which every finite semigroup has an idempotent.
	The proof is short, so we give it here for the sake of self-containedness.
	
	Because $\gen{x}_H$ is finite, there exist $n, k \in \NN^+$ such that $x^n = x^{n+k}$, and by induction this implies that $x^n = x^{n+hk}$ for all $h \in \NN$. Therefore, we find that
	\[
	(x^{nk})^2 = x^{2nk} = x^{(k+1)n}x^{(k-1)n} = x^n x^{(k-1)n} = x^{nk}.
	\]
	But $H$ has no non-trivial idempotents, thus it must be the case that $x^{nk}=1_H$. That is, $x$ is a unit of $H$, and we have $x^{-1} = x^{nk-1} \in \gen{x}_H$. So, $\gen{x}_H$ is a (finite) cyclic group of order $\ge 3$.
\end{proof}

%

\begin{thm}\label{th:atomicity}
	Let $H$ be a monoid. Then $\mathcal P_\fun(H)$ is atomic if and only if $1_H \ne x^2 \ne x$ for every $x \in H \setminus \{1_H\}$.
\end{thm}
%
\begin{proof}
	The ``only if'' part is a consequence of Lemma \ref{lem:2-elt-atoms}\ref{it:lem:no-non-id-elts-of-small-order-implies-structure(ii)}.
	As for the other direction, assume that $1_H \ne x^2 \ne x$ for each $x\in H\setminus\{1_H\}$, and fix $X \in \mathcal P_{\fun}(H)$ with $|X| \ge 2$.
	We wish to show that 
	\[
	X = A_1 \cdots A_n, \quad\text{for some }A_1, \ldots, A_n \in \mathcal{A}(\mathcal P_{\fin,1}(H)).
	\]
	If $X$ is a $2$-element set, the claim is true by Lemma \ref{lem:2-elt-atoms}\ref{it:lem:no-non-id-elts-of-small-order-implies-structure(i)}. So let $|X| \ge 3$, and suppose inductively that every $Y \in \P_\fun(H)$ with $2 \le |Y| < |X|$ is a product of atoms. If $X$ is an atom, we are done.
	Otherwise, $X = A B$ for some non-units $A, B \in \P_\fun(H)$, and by symmetry we can assume $|X| \ge |A| \ge |B| \ge 2$. 
	
	If $|A| < |X|$, then both $A$ and $B$ factor into a product of atoms (by the inductive hypothesis), and so too does $X=A B$. Consequently, we are only left to consider the case when $|X| = |A|$. 
	
	For, we notice that 
	$
	A \cup B \subseteq A B = X 
	$
	(because $1_H \in A \cap B$), and this is only possible if $A = X$ (since $|A| = |X|$ and $A \subseteq X$). So, to summarize, we have that
	\begin{equation}\label{equ:containments}
	|X| \ge 3, \quad |B| \ge 2, \quad\text{and}\quad B \subseteq AB = X = A.
	\end{equation}
	In particular, since $B$ is not a unit of $\mathcal P_{\fin,1}(H)$, we can choose an element $b\in B\setminus\{1_H\} \subseteq A$. Hence, taking $A_b := A \setminus \{b\}$, we have $|A_b| < |A|$, and it is easy to check that $A_b B = A = X$ (in fact, $1_H$ is in $A_b \cap B$, and therefore we derive from \eqref{equ:containments} that $
	A_b B \subseteq A = A_b \cup \{b\} \subseteq A_b B \cup \{b\} \subseteq A_b B \cup B = A_b B$).
	
	If $|B|<|A|$, then we are done, because $A_b$ and $B$ are both products of atoms (by the inductive hypothesis), and thus so is $X = AB = A_bB$.
	Otherwise, it follows from \eqref{equ:containments} and the above that 
	\begin{equation}\label{equ:further-identities}
	X = A = B = A_b B \quad\text{and}\quad |A| \ge 3, 
	\end{equation}
	so we can choose an element $a \in A \setminus\{1_H,b\}$. Accordingly, set $B_a := B \setminus \{a\}$. Then $|B_a| < |B|$ (because $A = B$ and $a \in A$), and both $A_b$ and $B_a$ decompose into a product of atoms (again by induction). But this finishes the proof, since it is straightforward from \eqref{equ:further-identities} that $X = A = A_b B_a$ (indeed, $1_H \in A_b \cap B_a$ and $b \in B_a$, so we find that $
	A_b B_b \subseteq A = A_b \cup \{b\} \subseteq A_b B_a \cup \{b\} \subseteq A_b B_a \cup B_a = A_b B_a$).
\end{proof}

Now with Proposition \ref{prop:equimorphism} and Theorem \ref{th:atomicity} in hand, we can engage in a finer study of the arithmetic of power monoids; in particular, we may wish to study their (systems of) sets of lengths. However, we are immediately met with a ``problem'' (i.e., some sets of lengths are infinite in a rather trivial way):
%
\begin{eg}\label{unbounded-fzn}
	Let $H$ be a monoid with an element $x$ of finite odd order $m \ge 3$, and set $X := \{x^k : k \in \NN \}$. Then it is clear that $X$ is the setwise product of $n$ copies of $\{1_H, x\}$ for every $n \ge m$. This shows that the set of lengths of $X$ relative to $\P_\fun(H)$ contains $\llb m, \infty \rrb$ (and hence is infinite), since we know from Lemma \ref{lem:2-elt-atoms} that $\{1_H, x\}$ is an atom of $\P_\fun(H)$.
\end{eg}
%
The nature of this problem is better clarified by our next result, and we will more thoroughly address it in \S{ }\ref{sec:minimal-factorizations}.
%
\begin{thm}\label{thm:BF-torsion}
	Let $H$ be a monoid. The following hold:
	\begin{enumerate}[label={\rm (\roman{*})}]
		\item\label{it:thm:BF-torsion(i)} If $H$ is torsion-free and $X\in \P_\fun(H)$, then $\sup \mathsf{L}_{\mathcal P_\fun(H)}(X) \le |X|^2-|X|$.
		\item\label{it:thm:BF-torsion(ii)} $\P_\fun(H)$ is \textup{BF} if and only if $H$ is torsion-free.
		\item\label{it:thm:BF-torsion(iii)} $\P_\funt(H)$ is \textup{BF} if and only if $H$ is torsion-free.
	\end{enumerate}
\end{thm}
\begin{proof}
	\ref{it:thm:BF-torsion(i)}
	Set $n := |X| \in \NN^+$, let $\ell$ be an integer $> (n-1)n$, and suppose for a contradiction that $X = A_1\cdots A_\ell$ for some $A_1,\dots, A_\ell\in \mathcal{A}(\P_\fun(H))$.
	By the Pigeonhole Principle, there is an element $x\in X$ and a subset $I \subseteq \llb 1,\ell \rrb$ such that $m := |I| \ge n$ and $x \in A_i$ for each $i\in I$.
	So, writing $I = \{i_1,\ldots, i_m\}$, we find that $x^k \in A_{i_1} \cdots A_{i_k} \subseteq A_1\cdots A_\ell = X$ for every $k \in \llb 1, m \rrb$, i.e., $\{1_H,x,\dots, x^m\} \subseteq X$.
	However, since $H$ is torsion-free, each power of $x$ is distinct, and hence $n = |X| \ge m+1 > n$ (a contradiction).
	
	\ref{it:thm:BF-torsion(ii)}
	First suppose for a contradiction that $\P_\fun(H)$ is BF and has an element $x$ of finite order $m$; then $\P_\fun(H)$ is also atomic, and we know by Theorem \ref{th:atomicity} and Lemma \ref{lem:no-non-id-elts-of-small-order-implies-structure}\ref{it:lem:no-non-id-elts-of-small-order-implies-structure(ii)} that $x^m = 1_H$.
	If $m$ is even then $(x^{m/2})^2 = 1_H$, contradicting the atomicity of $\P_\fun(H)$ since, by Theorem \ref{th:atomicity}, no non-identity element of $H$ can have order 2. 
	If $m$ is odd then Example \ref{unbounded-fzn} shows us that the set of lengths of $\{x^k : k \in \NN \}$ is infinite, a contradiction to the assumption that $\P_\fun(H)$ is BF.
	
	Conversely, suppose $H$ is torsion-free; then all powers of non-identity elements are distinct, so Theorem \ref{th:atomicity} implies that $\P_\fun(H)$ is atomic, and \ref{it:thm:BF-torsion(i)} gives an explicit upper bound on the lengths of factorizations.
	
	\ref{it:thm:BF-torsion(iii)} 
	By part \ref{it:thm:BF-torsion(ii)}, it is sufficient to show that $\P_\funt(H)$ is BF if and only if $\P_\fun(H)$ is BF.
	The ``only if'' direction follows from \cite[Theorem 2.28(iv) and Corollary 2.29]{fan-tringali18}, so suppose that $\P_\fun(H)$ is BF.
	Then $\P_\fun(H)$ is atomic, and hence, by Theorem \ref{th:atomicity}, $1_H \neq x^2 \neq x$ for all $x\in H\setminus\{1_H\}$.
	In view of Lemma \ref{lem:no-non-id-elts-of-small-order-implies-structure}\ref{it:lem:no-non-id-elts-of-small-order-implies-structure(ii)}, this implies that $H$ is Dedekind-finite, so the natural embedding $\P_\fun(H)\hookrightarrow\P_\funt(H)$ is an essentially surjective equimorphism by Proposition \ref{prop:funt&fun-have-the-same-system-of-lengths}\ref{it:prop:funt&fun-have-the-same-system-of-lengths(i)}.
	The result then follows from Proposition  \ref{prop:funt&fun-have-the-same-system-of-lengths}\ref{it:prop:funt&fun-have-the-same-system-of-lengths(iv)}.
\end{proof}

\section{Minimal factorizations and conditions for bounded minimal lengths} \label{sec:min-fac}
\label{sec:minimal-factorizations}
%

%
Example \ref{unbounded-fzn} seems to indicate that, in the presence of torsion in the ground monoid $H$, the sets of lengths in $\P_\fun(H)$ can blow up in a predictable fashion.
In the commutative setting, we could counteract such phenomena directly by considering only factorizations involving ``sufficiently low'' powers of atoms (cf. the notion of ``index'' and the corresponding sets of factorization classes defined in \cite{geroldinger-lettl90}).
We strive instead to axiomatize an approach which responds to \textit{all} non-cancellative phenomena in a general monoid, spurring us to introduce a refinement of our notion of factorization (recall that, given a set $X$, we let $\varepsilon_X$ be the identity of $\mathcal{F}^\ast(X)$, the free monoid with basis $X$).
%
\begin{defn}\label{def:preorder}
	Let $H$ be a monoid. We denote by $\preceq_H$ 
	the binary relation on $\mathcal{F}^\ast(\mathcal{A}(H))$ determined by taking $\mathfrak a \preceq_H \mathfrak b$, for some  
	$\mathcal{A}(H)$-words $\mathfrak a$ and $\mathfrak b$ of length $h$ and $k$ respectively, if and only if 
	\begin{enumerate}[label=$\bullet$]
		\item $\mathfrak b = \varepsilon_{\mathcal{A}(H)}$ and $\pi_H(\mathfrak a) = 1_H$, or
		\item $\mathfrak a$ and $\mathfrak b$ are non-empty words, say $\mathfrak a = a_1 \ast \cdots \ast a_h$ and $\mathfrak b = b_1 \ast \cdots \ast b_k$, and
		there is an injection $\sigma: \llb 1, h \rrb \to \llb 1, k \rrb$ such that $b_i \simeq_H a_{\sigma(i)}$ for every $i \in \llb 1, h \rrb$. 
	\end{enumerate}
	We shall write $\mathfrak{a} \prec_H \mathfrak{b}$ if $\mathfrak{a} \preceq_H \mathfrak{b}$ but $\mathfrak{b} \not\preceq_H \mathfrak{a}$, and say that a word $\mathfrak{a} \in \mathcal{F}^\ast(\mathcal{A}(H))$ is \textit{$\preceq_H$-minimal} (or simply \textit{minimal}) if there does not exist any $\A(H)$-word $\mathfrak{b}$ such that $\mathfrak{b} \prec_H \mathfrak{a}$.
\end{defn}
%
The next result highlights a few basic properties of the relation introduced in Definition \ref{def:preorder}.
%
\begin{prop}\label{preorder-facts}
	Let $H$ be a monoid, and let $\mathfrak{a},\mathfrak{b}\in \mathcal{F}^*(\A(H))$. The following hold:
	\begin{enumerate}[label = {\textup{(\roman{*})}}]
		\item\label{it:prop:preorder-facts(0)} $\preceq_H$ is a preorder \textup{(}i.e., a reflexive and transitive binary relation\textup{)} on $\mathcal{F}^\ast(\mathcal{A}(H))$.
		\item\label{it:prop:preorder-facts(ii)} If $\mathfrak{a} \preceq_H \mathfrak{b}$ then $\| \mathfrak{a} \|_H \le \| \mathfrak{b} \|_H$.
		\item\label{it:prop:preorder-facts(i)} $\varepsilon_{\mathcal{A}(H)} \preceq_H \mathfrak a$ if and only if $\mathfrak a = \varepsilon_{\mathcal{A}(H)}$.
		%\item\label{it:prop:preorder-facts(iii)} $\mathfrak{a} \preceq \mathfrak{b}$ if and only if $\mathfrak{b}$ has a subword $\mathfrak{b}' \in \llb \mathfrak{a} \rrb_{\mathcal{C}_H}$.
		\item\label{it:prop:preorder-facts(iv)} $\mathfrak{a} \preceq_H \mathfrak{b}$ and $\mathfrak{b} \preceq_H \mathfrak{a}$ if and only if $\mathfrak a \preceq_H \mathfrak b$ and $\|\mathfrak a\|_H = \|\mathfrak b\|_H$, if and only if $(\mathfrak{a},\mathfrak{b}) \in \C_H$.
	\end{enumerate}
\end{prop}
%
\begin{proof}
	Points \ref{it:prop:preorder-facts(0)} and \ref{it:prop:preorder-facts(ii)} are straightforward from our definitions, and for  \ref{it:prop:preorder-facts(i)} it suffices to note that, by Proposition \ref{prop:unit-adjust}\ref{it:prop:unit-adjust(0)}, $\pi_H(\mathfrak a) = 1_H$ if and only if $\mathfrak a = \varepsilon_{\mathcal{A}(H)}$. 
	
	As for \ref{it:prop:preorder-facts(iv)}, set $h := \|\mathfrak a\|_H$ and $k := \|\mathfrak b\|_H$. 
	By part \ref{it:prop:preorder-facts(ii)}, $\mathfrak{a} \preceq_H \mathfrak{b}$ and $\mathfrak{b} \preceq_H \mathfrak{a}$ only if  $\mathfrak a \preceq_H \mathfrak b$ and $h = k$; and it is immediate to check that $(\mathfrak{a},\mathfrak{b}) \in \C_H$ implies $\mathfrak{a} \preceq_H \mathfrak{b}$ and $\mathfrak{b} \preceq_H \mathfrak{a}$. 
	
	So, to finish the proof, assume that $\mathfrak a \preceq_H \mathfrak b$ and $h = k$. We only need to show that $(\mathfrak a, \mathfrak b) \in \mathcal{C}_H$. For, we have (by definition) that $\mathfrak a \preceq_H \mathfrak b$ if and only if $\pi_H(\mathfrak a) = \pi_H(\mathfrak b)$ and there is an injection $\sigma: \llb 1, h \rrb \to \llb 1, k \rrb$ 
	such that $a_i \simeq_H b_{\sigma(i)}$ for every $i \in \llb 1, h \rrb$. But $\sigma$ is actually a bijection (because $h = k$), and we can thus conclude that $(\mathfrak a, \mathfrak b) \in \mathcal{C}_H$.
\end{proof}
%
\begin{defn}\label{min-fac}
	Let $H$ be a monoid and $x \in H$. An $H$-word $\mathfrak{a}$ is a \textit{$\preceq_H$-minimal factorization} of $x$, or simply a \textit{minimal factorization} of $x$ (in or relative to $H$), if $\mathfrak a \in \mathcal Z_H(x)$ and $\mathfrak a$ is $\preceq_H$-minimal.
	Accordingly, 
	\[
	\mathcal{Z}_{H}^\m(x) := \left\{ \mathfrak{a}\in \mathcal{Z}_H(x): \mathfrak{a} \textrm{ is $\preceq_H$-minimal} \right\}
	\quad\text{and}\quad
	\mathsf{Z}_{H}^\m(x):= \mathcal{Z}_{H}^\m(x)/\mathcal{C}_H
	\]
	shall denote, respectively, the set of $\preceq_H$-minimal factorizations and the set of $\preceq_H$-minimal factorization classes of $x$ (cf. the definitions from \S{ }\ref{subsec:factorizations}). In addition, we take 
	\[
	\mathsf{L}_{H}^\m(x) := \left\{ \|\mathfrak{a}\|_H : \mathfrak{a} \in \mathcal{Z}_{H}^\m(x) \right\} \subseteq \NN
	\]
	to be the set of \textit{$\preceq_H$-minimal factorization lengths} of $x$, and
	\[
	\mathcal{L}^\m(H) := \left\{ \mathsf{L}_{H}^\m(x) : x\in H \right\} \subseteq \mathcal P(\NN)
	\]
	to be the \textit{system of sets of $\preceq_H$-minimal lengths} 
	of $H$.
	Lastly, we say that the monoid $H$ is
	\begin{itemize}
		\item \textup{BmF} 
		or \textit{bounded-minimally-factorial} (respectively, \textup{FmF} or \textit{finite-minimally-factorial}) if $\mathsf{L}_{H}^\m(x)$ (respectively, $\mathsf{Z}_{H}^\m(x)$) is finite and non-empty for every $x\in H\setminus H^\times$;
		\item \textup{HmF} or \textit{half-minimally-factorial} (respectively, \textit{minimally factorial}) if $\mathsf{L}_{H}^\m(x)$ (respectively, $\mathsf{Z}_{H}^\m(x)$) is a singleton for all $x \in H \setminus H^\times$.
		%
		%\item \textit{minimally factorial} if $\mathsf{Z}_{H}^\m(x)$ is a singleton
	\end{itemize}
	Note that we may write $\mathcal{Z}^\m(x)$ for $\mathcal{Z}_{H}^\m(x)$, $\mathsf {L}^\m(x)$ for $\mathsf {L}_{H}^\m(x)$, etc. if there is no likelihood of confusion.
\end{defn}
%
It is helpful, at this juncture, to observe some fundamental features of minimal factorizations.
%
\begin{prop}\label{prop:min-basics}
	Let $H$ be a monoid and let $x\in H$. The following hold:
	%
	\begin{enumerate}[label = {\rm (\roman{*})}]
		\item\label{it:prop:min-basics(i)} Any $\A(H)$-word of length $0$, $1$, or $2$ is minimal.
		\item\label{it:prop:min-basics(ii)} $\Z_H(x) \ne \emptyset$ if and only if  $\Z_H^\m(x) \ne \emptyset$.
		%
		\item\label{it:prop:min-basics(iib)} If $\mathfrak a \in \mathcal Z_H^\m(x)$ and $(\mathfrak a, \mathfrak b) \in \mathcal{C}_H$, then $\mathfrak b \in \mathcal Z_H^\m(x)$.
		%
		\item\label{it:prop:min-basics(iii)} If $K$ is a divisor-closed submonoid of $H$ and $x \in K$, then $\Z_K^\m(x) = \Z_H^\m(x)$ and $\mathsf L_K^\m(x) = \mathsf L_H^\m(x)$.
		\item\label{it:prop:min-basics(iv)} If $H$ is commutative and unit-cancellative, then $\Z_H^\m(x) = \Z_H(x)$, and hence $\mathsf L_H^\m(x) = \mathsf L_H(x)$.
	\end{enumerate}
\end{prop}
%
\begin{proof}
	\ref{it:prop:min-basics(i)}, \ref{it:prop:min-basics(ii)}, and \ref{it:prop:min-basics(iib)} are an immediate consequence of parts \ref{it:prop:preorder-facts(ii)}-\ref{it:prop:preorder-facts(iv)} of Proposition \ref{preorder-facts} (in particular, note that, if $\mathfrak a$  is an $\mathcal{A}(H)$-word of length $1$, then $\pi_H(\mathfrak a)$ is an atom of $H$, and therefore $\pi_H(\mathfrak a) \ne \pi_H(\mathfrak b)$ for every  $\mathcal{A}(H)$-words $\mathfrak b$ of length $\ge 2$); and \ref{it:prop:min-basics(iii)} follows at once from considering that, if $K$ is a divisor-closed submonoid of $H$ and $x \in K$, then $\mathcal Z_K(x) = \mathcal Z_H(x)$ and $\mathsf L_K(x) = \mathsf L_H(x)$, see \cite[Proposition 2.21(ii)]{fan-tringali18}.
	
	\ref{it:prop:min-basics(iv)} Assume $H$ is commutative and unit-cancellative. It suffices to check that no non-empty $\mathcal{A}(H)$-word has a proper subword with the same product.
	For, suppose to the contrary that there exist  $a_1,\dots,a_n \in \mathcal{A}(H)$ with $\prod_{i \in I} a_i = a_1\cdots a_n$ for some $I \subsetneq \llb 1, n \rrb$. Since $H$ is commutative, we can assume without loss of generality that $I = \llb 1, k \rrb$ for some $k \in \llb 0, n-1 \rrb$. Then unit-cancellativity implies $a_{k+1}\cdots a_n \in H^\times$, and we get from \cite[Proposition 2.30]{fan-tringali18} that $a_{k+1},\dots,a_n \in H^\times$, which is however impossible (by definition of an atom).	
	%	
	%	\ref{it:prop:min-basics(vi)}
\end{proof}
%
To further elucidate the behavior of minimal factorizations, we give an analogue of Proposition \ref{prop:unit-adjust}\ref{it:prop:unit-adjust(ii)} showing that multiplying a non-unit by units does not change its set of minimal factorizations.
%
\begin{lemma}\label{lem:min-unit-adjust}
	Let $H$ be a monoid, and fix $x\in H\setminus H^\times$ and $u,v\in H^\times$.
	Then there is a length-preserving bijection $\mathcal{Z}_H^\m(x)\to\mathcal{Z}_H^\m(uxv)$, and in particular $\mathsf L_H^\m(x) = \mathsf L_H^\m(uxv)$.
\end{lemma}
\begin{proof}
	Given $w, z \in H$ and a non-empty word $\mathfrak{z} = y_1 \ast \cdots \ast y_n \in \mathcal{F}^\ast(H)$ of length $n$, denote by $w\mathfrak{z}z$ the length-$n$ word $\bar{y}_1 \ast \cdots \ast \bar{y}_n \in \mathcal{F}^\ast(H)$ defined by taking $\bar{y}_1 := w y_1 z$ if $n = 1$, and $\bar{y}_1:= wy_1$, $\bar{y}_n := y_nz$, and $\bar{y}_i := y_i$ for all $i\in \llb 2,n-1\rrb$ otherwise. 
	We claim that the function 
	$$
	f: \mathcal{Z}_H^\m(x) \to \mathcal{Z}_H^\m(uxv): \mathfrak a \mapsto u\mathfrak a v
	$$
	is a well-defined length-preserving bijection. 
	In fact, it is sufficient to show that $f$ is well-defined, since this will in turn imply that the map $g: \mathcal{Z}_H^\m(uxv)\to\mathcal{Z}_H^\m(x):\mathfrak{b}\mapsto u^{-1}\mathfrak{b}v^{-1}$ is also well-defined (observe that $uxv \in H \setminus H^\times$ and $x = u^{-1} uxv v^{-1}$), and then it is easy to check that $g$ is the inverse of $f$.
	
	For the claim, let $\mathfrak a \in \mathcal{Z}_H^\m(x)$, and note that, by Proposition \ref{prop:unit-adjust}\ref{it:prop:unit-adjust(0)}, $\|\mathfrak a\|_H$ is a positive integer, so that  $\mathfrak a = a_1 \ast \cdots \ast a_n$ for some $a_1, \ldots, a_n \in \mathcal{A}(H)$. 
	In view of Proposition \ref{prop:unit-adjust}\ref{it:prop:unit-adjust(i)}, $u\mathfrak{a}v$ is a factorization of $uxv$, and we only need to verify that it is also $\preceq_H$-minimal. For,
	suppose to the contrary that $\mathfrak b \prec_H u\mathfrak av$ for some $\mathfrak b \in \mathcal{F}^\ast(\mathcal{A}(H))$.
	Then $\pi_H(\mathfrak b) = \pi_H(u\mathfrak a v) = uxv$ and, by Proposition \ref{preorder-facts}\ref{it:prop:preorder-facts(iv)},
	$
	k := \|\mathfrak b\|_H \in \llb 1, n-1 \rrb
	$
	(recall that $uxv \notin H^\times$). So, $\mathfrak b = b_1 \ast \cdots \ast b_k$ for some atoms  $b_1, \ldots, b_k \in H$, and there exists an injection $\sigma: \llb 1, k \rrb \to \llb 1, n \rrb$ such that $b_i \simeq_H a_{\sigma(i)}$ for each $i \in \llb 1, k \rrb$.
	Define
	$
	\mathfrak{c} := u^{-1} \mathfrak b v^{-1}$. 
	
	By construction and Proposition \ref{prop:unit-adjust}\ref{it:prop:unit-adjust(i)}, there are  $c_1, \ldots, c_k \in \mathcal{A}(H)$ such that $\mathfrak c = c_1 \ast \cdots \ast c_k$; and it follows from the above that $\pi_H(\mathfrak c) = u^{-1} \pi_H(\mathfrak b) v^{-1} = x$ and $c_i \simeq_H a_{\sigma(i)}$ for every $i \in \llb 1, k \rrb$. Since $k < n$, we can thus conclude from Proposition \ref{preorder-facts}\ref{it:prop:preorder-facts(iv)} that $\mathfrak c \prec_H \mathfrak a$, contradicting the $\preceq_H$-minimality of $\mathfrak{a}$.
\end{proof}
%
We saw in the previous section that equimorphisms transfer factorizations between monoids (Proposition \ref{prop:equimorphism}).  
Equimorphisms have a similar compatibility with minimal factorizations, in the sense that an equimorphism also satisfies a ``minimal version'' of condition \ref{def:equimorphism(E3)} from Definition \ref{def:equimorphism}.
%
%
%
\begin{prop}\label{prop:min-equi}
	Let $H$ and $K$ be monoids and $\varphi: H\to K$ an equimorphism. The following hold: 
	\begin{enumerate}[label={\rm (\roman{*})}]
		\item\label{it:prop:min-equi(i)} If $x\in H \setminus H^\times$ and $\mathfrak{b}\in \mathcal{Z}_K^\m(\varphi(x))$, then there is $\mathfrak{a}\in \mathcal{Z}_H^\m(x)$ with $\varphi^*(\mathfrak{a})\in \llb \mathfrak{b} \rrb_{\mathcal{C}_K}$.
		%
		\item\label{it:prop:min-equi(ii)} $\mathsf{L}_K^\m(\varphi(x)) \subseteq \mathsf{L}_H^\m(x)$ for every $x\in H\setminus H^\times$.
		\item\label{it:prop:min-equi(iii)} If $\varphi$ is essentially surjective then, for all $y\in K\setminus K^\times$, there is $x\in H\setminus H^\times$ with $\mathsf{L}_K^\m(y) \subseteq \mathsf{L}_H^\m(x)$. 
	\end{enumerate}
\end{prop}
%
\begin{proof}
	\ref{it:prop:min-equi(i)}
	Pick $x\in H \setminus H^\times$, and let $\mathfrak b \in \mathcal{Z}_K^\m(\varphi(x))$. Then $\mathfrak b \ne \varepsilon_{\mathcal{A}(K)}$, otherwise $\varphi(x) = \pi_K(\mathfrak b) = 1_K$ and, by \ref{def:equimorphism(E1)}, $x \in \varphi^{-1}(\varphi(x)) = \varphi^{-1}(1_K) \subseteq H^\times$ (a contradiction). Consequently, \ref{def:equimorphism(E3)} yields the existence of a factorization $\mathfrak{a}\in \mathcal{Z}_H(x)$ with $\varphi^*(\mathfrak{a}) \in \llb \mathfrak{b} \rrb_{\mathcal{C}_K}$, and it only remains to show that $\mathfrak a$ is $\preceq_H$-minimal. 
	
	For, note that $n := \|\mathfrak a\|_H = \|\varphi^\ast(\mathfrak a)\|_K = \|\mathfrak b\|_K \ge 1$, and write $\mathfrak a = a_1 \ast \cdots \ast a_n$ and $\mathfrak b = b_1 \ast \cdots \ast b_n$, with $a_1, \ldots, a_n \in \mathcal{A}(H)$ and $b_1, \ldots, b_n \in \mathcal{A}(K)$. Then suppose to the contrary that $\mathfrak a$ is not $\preceq_H$-minimal, i.e., there exist a (necessarily non-empty) $\mathcal{A}(H)$-word $\mathfrak{c} = c_1 \ast \cdots \ast c_m$ and an injection $\sigma: \llb 1, m \rrb \to \llb 1, n \rrb$ such that $\pi_H(\mathfrak c) = \pi_H(\mathfrak a) = x$ and $c_i \simeq_H a_{\sigma(i)}$ for every $i \in \llb 1, m \rrb$. Then
	\[
	\pi_K(\varphi^\ast(\mathfrak c)) = \varphi(c_1) \cdots \varphi(c_m) = \varphi(x)
	\quad\text{and}\quad
	\varphi(c_1) \simeq_K \varphi(a_{\sigma(1)}), \ldots, \varphi(c_m) \simeq_K \varphi(a_{\sigma(m)})
	\]
	(recall that monoid hom\-o\-mor\-phisms map units to units; so, if $u \simeq_H v$, then $\varphi(u) \simeq_K \varphi(v)$); and together with Proposition \ref{prop:min-basics}\ref{it:prop:min-basics(iib)}, this proves that 
	$\varphi^\ast(\mathfrak c) \prec_K \mathfrak b$, contradicting the $\preceq_K$-minimality of $\mathfrak b$.
	
	\ref{it:prop:min-equi(ii)} Fix $x \in H \setminus H^\times$, and suppose $\mathsf{L}_K^\m(\varphi(x)) \ne \emptyset$ (otherwise there is nothing to prove). Accordingly, let $k \in \mathsf{L}_K^\m(\varphi(x))$ and $\mathfrak b \in \mathcal Z_K^\m(\varphi(x))$ such that $k = \|\mathfrak b\|_K$. It is sufficient to check that $k \in \mathsf L_H^\m(x)$, and this is straightforward: Indeed, we have by \ref{it:prop:min-equi(i)} that $\varphi^\ast(\mathfrak a)$ is $\mathcal{C}_K$-congruent to $\mathfrak b$ for some $\mathfrak a \in \mathcal Z_H^\m(x)$, which implies in particular that $k = \|\varphi^\ast(\mathfrak a)\|_K = \|\mathfrak a\|_H \in \mathsf L_H^\m(x)$.
	
	\ref{it:prop:min-equi(iii)} Assume $\varphi$ is essentially surjective, and let $y \in K \setminus K^\times$. Then $y = u \varphi(x) v$ for some $u,v\in K^\times$ and $x\in H$, and neither $x$ is a unit of $H$ nor $\varphi(x)$ is a unit of $K$ (because $\varphi(H^\times) \subseteq K^\times$ and $y \notin K^\times$). Accordingly, we have by Lemma \ref{lem:min-unit-adjust} and part \ref{it:prop:min-equi(ii)} that $\mathsf{L}_K^\m(y) = \mathsf{L}_K^\m(\varphi(x)) \subseteq \mathsf{L}_H^\m(x)$.
\end{proof}
%
Let $H$ be a monoid. As in \S{ }\ref{sec:atomicity}, we would like to simplify the study of minimal factorizations in $\P_\funt(H)$ as much as possible by passing to consideration of the reduced monoid $\P_\fun(H)$. For, we have to make clear the nature of the relationship between minimal factorizations in $\P_\funt(H)$ and those in $\P_\fun(H)$.
We shall see that this is possible under \textit{some} circumstances. 
%
\begin{prop}\label{prop:comm-pm}
	Let $H$ be a commutative monoid, and let $X\in \P_{\fin,1}(H)$. The following hold:
	\begin{enumerate}[label = {\rm (\roman{*})}]
		\item\label{it:prop:comm-pm(i)} $\Z_{\P_{\fin,1}(H)}^\m(X) \subseteq \Z_{\P_{\funt}(H)}^\m(X)$.
		\item\label{it:prop:comm-pm(ii)} $\mathsf{L}_{\P_{\fin,1}(H)}^\m(X) = \mathsf{L}_{\P_\funt(H)}^\m(X)$.
		\item\label{it:prop:comm-pm(iii)} $\mathcal{L}^\m(\P_{\fin,1}(H)) = \mathcal{L}^\m(\P_\funt(H))$.
	\end{enumerate}
\end{prop}
%
\begin{proof}
	\ref{it:prop:comm-pm(i)} Let $\mathfrak{a}$ be a minimal factorization of $X$ relative to $\P_{\fin,1}(H)$. In light of Proposition \ref{prop:min-basics}\ref{it:prop:min-basics(i)}, $\mathfrak{a}$ is a non-empty $\mathcal{A}(\mathcal P_{\fin,1}(H))$-word, i.e., $\mathfrak a = A_1 \ast \cdots \ast A_n$ for some atoms $A_1, \ldots, A_n \in \mathcal P_{\fin,1}(H)$. 
	
	Assume for the sake of contradiction that $\mathfrak a$ is not a minimal factorization relative to $\P_{\fin,\times}(H)$. Then there exist a non-empty $\mathcal{A}(\mathcal P_{\fin,\times}(H))$-word $\mathfrak b = B_1 * \cdots * B_m$ and an injection $\sigma: \llb 1, m \rrb \to \llb 1, n \rrb$ with 
	\[
	X = A_1 \cdots A_n = B_1 \cdots B_m
	\quad\text{and}\quad
	B_1 \simeq_{\mathcal P_{\fin,\times}(H)} A_{\sigma(1)}, \ldots, B_m \simeq_{\mathcal P_{\fin,\times}(H)} A_{\sigma(m)},
	\]
	and on account of Proposition \ref{preorder-facts}\ref{it:prop:preorder-facts(iv)} we must have $1 \le m < n$.
	Since $H$ is a commutative monoid, this means in particular that, for each $i \in \llb 1, m \rrb$, there is $u_i \in H^\times$ such that $B_i = u_i A_{\sigma(i)}$. Thus we have
	\[
	A_1 \cdots A_n = B_1 \cdots B_m = (u_1 A_{\sigma(1)}) \cdots (u_m A_{\sigma(m)}) = u \cdot A_{\sigma(1)} \cdots A_{\sigma(m)},
	\]
	where $u := u_1 \cdots u_m \in H^\times$. In view of Proposition \ref{prop:pm-arith}\ref{it:prop:pm-arith(ii)}, it follows that
	\[
	\left| A_1 \cdots A_n \right| = \left|A_{\sigma(1)} \cdots A_{\sigma(m)} \right|,
	\]
	which is only possible if 
	\[
	X = A_1 \cdots A_n = A_{\sigma(1)} \cdots A_{\sigma(m)}, 
	\]
	because $1_H \in A_i$ for every $i \in \llb 1, n \rrb$, and hence $A_{\sigma(1)} \cdots A_{\sigma(m)} \subseteq A_1 \cdots A_n$ (note that here we use again that $H$ is commutative). So, letting $\mathfrak a^\prime$ be the $\mathcal{A}(\mathcal P_{\fin,1}(H))$-word $ A_{\sigma(1)} \ast \cdots \ast A_{\sigma(m)}$ and recalling from the above that $m \le n-1$, we see by Proposition \ref{preorder-facts}\ref{it:prop:preorder-facts(iv)} that $\mathfrak a^\prime \prec_{\mathcal P_{\fin,1}(H)} \mathfrak a$, which contradicts the hypothesis that $\mathfrak a$ is a minimal factorization of $X$ in $\mathcal P_{\fin,1}(H)$.
	
	\ref{it:prop:comm-pm(ii)} It is an immediate consequence of part \ref{it:prop:comm-pm(i)} and Propositions \ref{prop:funt&fun-have-the-same-system-of-lengths}\ref{it:prop:funt&fun-have-the-same-system-of-lengths(i)} and \ref{prop:min-equi}\ref{it:prop:min-equi(ii)}, when considering that every commutative monoid is Dedekind-finite.
	
	\ref{it:prop:comm-pm(iii)} We already know from part \ref{it:prop:comm-pm(ii)} that $\mathcal{L}^\m(\mathcal P_{\fin,1}(H)) \subseteq \mathcal{L}^\m(\mathcal P_\funt(H))$. For the opposite inclusion, fix $X \in \P_\funt(H)$. We claim that there exists $Y \in \mathcal P_{\fin,1}(H)$ with 
	$\mathsf L_{\P_\funt(H)}^\m(X) = \mathsf L_{\P_{\fin,1}(H)}^\m(Y)$.
	Indeed, pick $x \in X \cap H^\times$. Then $x^{-1} X\in \P_{\fin,1}(H)$, and we derive from Lemma \ref{lem:min-unit-adjust} and part \ref{it:prop:comm-pm(ii)} that 
	\[
	\mathsf{L}_{\P_\funt(H)}^\m(X) = \mathsf{L}_{\P_\funt(H)}^\m(x^{-1}X) = \mathsf{L}_{\P_{\fin,1}(H)}^\m(x^{-1} X),
	\]
	which proves our claim and suffices to finish the proof (since $X$ was arbitrary).
\end{proof}
%
We will now discuss an instance in which equality in Proposition \ref{prop:comm-pm}\ref{it:prop:comm-pm(ii)} does not necessarily hold true in the absence of commutativity, and the best we can hope for is the containment relation implied by Proposition \ref{prop:min-equi}\ref{it:prop:min-equi(ii)} when $\varphi$ is the natural embedding of  Proposition \ref{prop:funt&fun-have-the-same-system-of-lengths}\ref{it:prop:funt&fun-have-the-same-system-of-lengths(i)}.
%is strict, so proving that the commutativity of $H$ in Proposition \ref{prop:min-equi} is critical.
%, by examining the natural embedding $\mathcal P_{\fin,1}(H) \hookrightarrow \mathcal P_{\fin,\times}(H)$, cf. Proposition \ref{prop:funt&fun-have-the-same-system-of-lengths}\ref{it:prop:funt&fun-have-the-same-system-of-lengths(i)}.
%
\begin{eg}\label{exa:strict-inclusion}
	Let $n$ be a (positive) multiple of $105$, and $p$ a (positive) prime dividing $n^2 + n + 1$; note that $p \ge 11$ and $3 \le n \bmod p \le p-3$ (where $n \bmod p$ is the smallest non-negative integer $\equiv r \bmod p$). Following \cite[p. 27]{gorenstein80}, we take $H$ to be the metacyclic group generated by the $2$-element set $\{r, s\}$ subject to $\ord_H(r) = p$, $\ord_H(s) = 3$, and $s^{-1} r s = r^n$. 
	Then $H$ is a non-abelian group of (odd) order $3p$, and by Theorem \ref{th:atomicity} and Propositions \ref{prop:equimorphism}\ref{it:prop:equimorphism(ii)} and \ref{prop:funt&fun-have-the-same-system-of-lengths}\ref{it:prop:funt&fun-have-the-same-system-of-lengths(i)}, $\mathcal P_{\fin,1}(H)$ and $\mathcal P_{\fin,\times}(H)$ are both atomic monoids. 
	
	We claim that $X := \langle r \rangle_H$ has minimal factorizations of length $p-1$ in $\mathcal P_{\fin,1}(H)$ but not in $\mathcal P_{\fin,\times}(H)$.
	For, pick $g \in X \setminus \{1_H\}$. Clearly $\ord_H(g) = p$, and thus we get from Lemma \ref{lem:2-elt-atoms}\ref{it:lem:2-elt-atoms(i)} that $\{1_H, g\}$ is an atom of $\mathcal P_{\fin,1}(H)$. Then it is immediate to see that $\mathfrak a_g := \{1_H, g\}^{\ast (p-1)}$ is a minimal factorization of $X$ in $\mathcal P_{\fin,1}(H)$; most notably, $\mathfrak a_g$ is minimal since otherwise there should exist an exponent $k \in \llb 1, p-2 \rrb$ such that $g^{p-1} = g^k$, contradicting that $\ord_H(g) = p$. Yet, $\mathfrak a_g$ is not a minimal factorization of $X$ in $\mathcal P_{\fin,\times}(H)$.
	Indeed, Proposition \ref{prop:funt&fun-have-the-same-system-of-lengths}\ref{it:prop:funt&fun-have-the-same-system-of-lengths(ii)} and Lemma \ref{lem:2-elt-atoms}\ref{it:lem:2-elt-atoms(i)} guarantee that $\{1_H, g\}$ and $\{1_H, g^n\}$ are associate atoms of $\mathcal P_{\fin,\times}(H)$, because $s^{-1} g^n s = g$ and, hence, $s^{-1} \{1, g\} s = \{1_H, g^n\}$. So, in view of Proposition \ref{preorder-facts}\ref{it:prop:preorder-facts(iv)}, it is straightforward that 
	\[
	\{1_H, g\}^{\ast (p-2)} \ast \{1_H, g^n\} \prec_{\mathcal P_{\fin,\times}(H)} \mathfrak a_g,
	\]
	In particular, note here that we have used that $3 \le n \bmod p \le p-3$ to obtain
	\[
	\{1_H, g, \ldots, g^{p-2}\} \cup \{g^n, g^{n+1}, \ldots, g^{n+p-2}\} = \{1_H, g, \ldots, g^{p-1}\} = X.
	\]
	Given that, suppose for a contradiction that $X$ has a minimal factorization $\mathfrak c$ of length $p-1$ in $\mathcal P_{\fin,\times}(H)$. Then by Propositions \ref{prop:funt&fun-have-the-same-system-of-lengths}\ref{it:prop:funt&fun-have-the-same-system-of-lengths(i)} and \ref{prop:min-equi}\ref{it:prop:min-equi(i)}, $\mathfrak c$ is $\mathcal{C}_{\mathcal P_{\fin,\times}(H)}$-congruent to a $\preceq_{\mathcal P_{\fin,1}(H)}$-minimal fac\-tor\-ization $\mathfrak a = A_1 \ast \cdots \ast A_{p-1}$ of $X$ of length $p-1$; and we aim to show that $\mathfrak a$ is $\mathcal{C}_{\mathcal P_{\fin,1}(H)}$-congruent to $\mathfrak a_g$ for some $g \in X \setminus \{1_H\}$, which is however impossible as it would mean that $\mathfrak a_g$ is a minimal factorization of $X$ in $\mathcal P_{\fin,\times}(H)$, in contradiction to what established in the above.
	
	Indeed, let $B_i$ be, for $i \in \llb 1, p-1 \rrb$, the image of $\{k \in \llb 0, p-1 \rrb: r^k \in A_i\} \subseteq \bf Z$ under the canonical map $\ZZ \to \ZZ/p\ZZ$. Then $\mathfrak a$ is a minimal factorization of $X$ in $\mathcal P_{\fin,1}(H)$ only if $\mathfrak b := B_1 \ast \cdots \ast B_{p-1}$ is a minimal factorization of $\ZZ/p\ZZ$ in the reduced power monoid of $(\ZZ/p\ZZ, +)$, herein denoted by $\mathcal P_{\fin,0}(\ZZ/p\ZZ)$.
	
	We want to show that $\mathfrak b$ is $\preceq_{\mathcal P_{\fin,0}(H)}$-minimal only if there is a non-zero $x \in \ZZ/p\ZZ$ such that $B_i = \{\overline{0}, x\}$ or $B_i = \{\overline{0}, -x\}$, or equivalently $A_i = \{1_H, r^{\hat{x}}\}$ or $A_i = \{1_H, r^{-\hat{x}}\}$, for every $i \in \llb 1, p-1 \rrb$ (for notation, see \S{ }\ref{subsec:generalities}). By the preceding arguments, this will suffice to conclude that $p-1 \notin \mathsf L^\m_{\mathcal P_\funt(H)}(X)$, because it implies at once that $\mathfrak a$ is $\mathcal C_{\mathcal P_\fun(H)}$-congruent to $\mathfrak a_g$ with $g := r^{\hat{x}} \in X \setminus \{1_H\}$.
	
	To begin, let $K$ be a subset of $\llb 1, p-1 \rrb$, and define $\mathcal{S}_K := \sum_{k \in K} B_k$ and $s_K := \{k \in K: |B_k| \ge 3\}$. Then we have by the Cauchy-Davenport inequality (see, e.g., \cite[Theorem 6.2]{grynkiewicz13}) that
	%
	\begin{equation}\label{equ:cauchy-davenport-application}
	\mathcal{S}_K = \ZZ/p\ZZ 
	\quad\text{or}\quad
	|\mathcal{S}_K| \ge 1 + {\sum}_{k \in K} \bigl(|B_k| - 1\bigr) \ge 1 + |K| + s_K.
	\end{equation}
	Now, let $I$ and $J$ be disjoint subsets of $\llb 1, p-1 \rrb$ with $|I \cup J| = |I| + |J| = p-2$. We claim $s_I = s_J = 0$. Indeed, it is clear that $\mathcal{S}_{I \cup J} \ne \ZZ/p\ZZ$, otherwise $\mathfrak b$ would not be a minimal factorization in $\mathcal P_{\fin,0}(\ZZ/p\ZZ)$. So, another application of the Cauchy-Davenport inequality, combined with \eqref{equ:cauchy-davenport-application}, yields
	%
	\begin{equation}\label{equ:another-cauchy-davenport-application}
	|S_{I \cup J}| = |S_I + S_J| \ge |S_I| + |S_J| - 1 \ge 1 + |I| + |J| + s_I + s_J = p-1 + s_I + s_J.
	\end{equation}
	%
	This suffices to prove that $|S_I + S_J| = p-1$ and $s_I = s_J = 0$, or else $S_{I \cup J} = \ZZ/p\ZZ$ (a contradiction). 
	
	It follows $|B_1| = \cdots = |B_{p-1}| = 2$. So, taking $I$ in \eqref{equ:another-cauchy-davenport-application} to range over all $1$-element subsets of $\llb 1, p-1 \rrb$ and observing that, consequently, $|S_J| \ge p-1-|S_I| = p-3 \ge 8 > |S_I|$, we infer from Vosper's theorem (see, e.g., \cite[Theorem 8.1]{grynkiewicz13}) that there exists a non-zero $x \in \ZZ/p\ZZ$ such that, for every $i \in \llb 1, p-1 \rrb$, $B_i$ is an arithmetic progression of $\ZZ/p\ZZ$ with difference $x$, i.e., $B_i = \{\overline{0}, x\}$ or $B_i = \{\overline{0}, -x\}$ (as wished).
\end{eg}



We proceed with an analogue of Theorem \ref{thm:BF-torsion}\ref{it:thm:BF-torsion(i)} and then prove the main results of the section.

\begin{prop}\label{prop:bounded-minimal-fzn}
	Let $H$ be a monoid and $X \in \P_\funt(H)$. The following hold:
	\begin{enumerate}[label={\rm (\roman{*})}]
		\item\label{it:prop:bounded-minimal-fzn(i)} If $X\in \P_\fun(H)$, then a minimal factorization of $X$ in $\P_\fun(H)$ has length $\le |X|-1$.
		\item\label{it:prop:bounded-minimal-fzn(ii)} If $H$ is Dedekind-finite, then a minimal factorization of $X$ in $\P_\funt(H)$ has length $\le |X|-1$.
	\end{enumerate}
\end{prop}
%
\begin{proof}
	\ref{it:prop:bounded-minimal-fzn(i)} The claim is trivial if $X = \{1_H\}$, when the only factorization of $X$ is the empty word; or if $X \in \mathcal{A}(\P_{\fin,1}(H))$, in which case $|X| \ge 2$ and $X$ has a unique factorization (of length $1$). So, assume that $X$ is neither the identity nor an atom of $\mathcal P_{\fin,1}(H)$, and let $\mathfrak a$ be a minimal factorization of $X$ (relative to $\P_{\fin,1}(H)$). Then $\mathfrak a = A_1*\cdots* A_n$, where $A_1, \ldots, A_n \in \mathcal{A}(\mathcal P_{\fin,1}(H))$ and $n \ge 2$; and we claim that
	\[
	A_1\cdots A_i \subsetneq A_1\cdots A_{i+1}, \quad \text{for every }i \in \llb 1, n-1 \rrb.
	\]
	In fact, let $i \in \llb 1, n-1 \rrb$. Since $1_H \in A_{i+1}$, it is clear that $A_1\cdots A_i \subsetneq A_1\cdots A_{i+1}$; and the inclusion must be strict, or else $A_1 \ast \cdots \ast A_i \ast \mathfrak b \prec_{\mathcal P_{\fin,1}(H)} \mathfrak a$, where $\mathfrak b := \varepsilon_{\mathcal{A}(\mathcal P_{\fin,1}(H))}$ if $i = n-1$ and $\mathfrak b := A_{i+2} \ast \cdots \ast A_n$ otherwise (contradicting the minimality of $\mathfrak a$). Consequently, we see that
	$
	2 \le |A_1\cdots A_i | < |A_1\cdots A_{i+1}| \le |X|$ for all $i \in \llb 1, n-1 \rrb$, and this implies at once that $n\le |X|-1$.
	
	\ref{it:prop:bounded-minimal-fzn(ii)} The conclusion is immediate from part \ref{it:prop:bounded-minimal-fzn(i)} and Propositions \ref{prop:funt&fun-have-the-same-system-of-lengths}\ref{it:prop:funt&fun-have-the-same-system-of-lengths(i)} and \ref{prop:min-equi}\ref{it:prop:min-equi(iii)}.
\end{proof}
%
\begin{thm}\label{BmF-char}
	Let $H$ be a monoid. Then the following are equivalent:
	%
	\begin{enumerate}[label={\rm (\alph{*})}]
		\item\label{it:BmF-char(a)} $1_H \ne x^2 \ne x$ for every $x \in H \setminus \{1_H\}$.
		\item\label{it:BmF-char(b)} $\P_\fun(H)$ is atomic.
		\item\label{it:BmF-char(c)} $\P_\fun(H)$ is \textup{BmF}.
		\item\label{it:BmF-char(d)} $\P_\fun(H)$ is \textup{FmF}.
		\item\label{it:BmF-char(e)} Every $2$-element subset $X$ of $H$ with $1_H \in X$ is an atom of $\mathcal P_\fun(H)$.
		\item\label{it:BmF-char(f)} $\P_\funt(H)$ is atomic.
		\item\label{it:BmF-char(g)} $\P_\funt(H)$ is \textup{BmF}.
		\item\label{it:BmF-char(h)} $\P_\funt(H)$ is \textup{FmF}.
		\item\label{it:BmF-char(i)} Every $2$-element subset $X$ of $H$ with $X \cap H^\times \ne \emptyset$ is an atom of $\mathcal P_\funt(H)$.
	\end{enumerate}
	%
\end{thm}
%
\begin{proof}
	We already know from Theorem \ref{th:atomicity} and Lemma \ref{lem:2-elt-atoms} that \ref{it:BmF-char(b)} $\Leftrightarrow$ \ref{it:BmF-char(a)} $\Leftrightarrow$ \ref{it:BmF-char(e)} and \ref{it:BmF-char(i)} $\Rightarrow$ \ref{it:BmF-char(a)}; while it is straightforward from our definitions that \ref{it:BmF-char(h)} $\Rightarrow$ \ref{it:BmF-char(g)} $\Rightarrow$ \ref{it:BmF-char(f)}. So, it will suffice to prove that \ref{it:BmF-char(b)}
	$\Rightarrow$ \ref{it:BmF-char(c)} $\Rightarrow$ \ref{it:BmF-char(d)} $\Rightarrow$ \ref{it:BmF-char(h)}
	and \ref{it:BmF-char(f)} $\Rightarrow$
	\ref{it:BmF-char(i)}.
	
	%	$$
	%	\text{\ref{it:BmF-char(b)} }
	%	\Rightarrow 
	%	\text{ \ref{it:BmF-char(c)} }
	%	\Rightarrow 
	%	\text{ \ref{it:BmF-char(d)} }
	%	\Rightarrow
	%	\text{ \ref{it:BmF-char(h)} }
	%	\quad\text{and}\quad
	%	\text{ \ref{it:BmF-char(f)} }
	%	\Rightarrow
	%	\text{ \ref{it:BmF-char(b)}}.
	%	$$
	\ref{it:BmF-char(b)} $\Rightarrow$ \ref{it:BmF-char(c)}: If $X \in \mathcal P_\fun(H)$ is a non-unit, then $\mathcal{Z}_{\mathcal P_\fun(H)}(X)$ is non-empty, and by Propositions \ref{prop:min-basics}\ref{it:prop:min-basics(ii)} and \ref{prop:bounded-minimal-fzn}\ref{it:prop:bounded-minimal-fzn(i)} we have that $\emptyset \ne \mathsf{L}_{\mathcal P_\fun(H)}^\m(X) \subseteq \llb 1, |X|-1\rrb$. So, $\mathcal P_\fun(H)$ is BmF.
	
	\ref{it:BmF-char(c)} $\Rightarrow$ \ref{it:BmF-char(d)}: Let $X \in \mathcal P_\fun(H)$ be a non-unit. 
	By Proposition \ref{prop:pm-arith}\ref{it:prop:pm-arith(i)}, any atom of $\mathcal P_\fun(H)$ dividing $X$ must be a subset of $X$, and there are only finitely many of these (since $X$ is finite).
	Because a minimal factorization of $X$ is a bounded $\mathcal{A}(\mathcal P_\fun(H))$-word (by the assumption that $H$ is BmF), it follows that $X$ has finitely many minimal factorizations, and hence $\mathcal P_\fun(H)$ is FmF (since $X$ was arbitrary).
	
	\ref{it:BmF-char(d)} $\Rightarrow$ \ref{it:BmF-char(h)}: Pick a non-unit $X \in \mathcal P_{\fin,\times}(H)$, and let $u \in H^\times$ such that $uX \in \mathcal P_{\fin,1}(H)$. Since $\mathcal P_{\fin,1}(H)$ is FmF (by hypothesis), it is also atomic. Hence, by Theorem \ref{th:atomicity} and Lemma \ref{lem:no-non-id-elts-of-small-order-implies-structure}\ref{it:lem:no-non-id-elts-of-small-order-implies-structure(i)}, $H$ is Dedekind-finite, and so we have by Proposition \ref{prop:funt&fun-have-the-same-system-of-lengths}\ref{it:prop:funt&fun-have-the-same-system-of-lengths(i)} that the natural embedding $\mathcal P_{\fin,1}(H) \hookrightarrow \mathcal P_{\fin,\times}(H)$ is an essentially surjective equimorphism.  
	In particular, we infer from Proposition \ref{prop:min-equi}\ref{it:prop:min-equi(i)} that any minimal factorization of $uX$ in $\mathcal P_{\fin,\times}(H)$ is $\mathcal{C}_{\mathcal P_{\fin,\times}(H)}$-congruent to a minimal factorization of $uX$ in $\mathcal P_{\fin,1}(H)$.
	However, this makes $\mathsf{Z}_{\mathcal P_{\fin,\times}(H)}^\m(uX)$ finite, whence $\mathsf{Z}_{\mathcal P_{\fin,\times}(H)}^\m(X)$ must also be finite as a consequence of Lemma \ref{lem:min-unit-adjust}.
	
	
	
	\ref{it:BmF-char(f)} $\Rightarrow$ \ref{it:BmF-char(i)}: Let $X$ be a $2$-element subset of $H$ with $X \cap H^\times \ne \emptyset$. Then $X = uA$ for some unit $u \in H^\times$, where $A := u^{-1}X$ is a $2$-element subset of $H$ with $1_H \in H$; and since $\mathcal P_{\fin,\times}(H)$ is atomic (by hypothesis), we are guaranteed by Lemmas \ref{lem:2-elt-atoms} and \ref{lem:no-non-id-elts-of-small-order-implies-structure}\ref{it:lem:no-non-id-elts-of-small-order-implies-structure(i)} that $A$ is an atom of $\mathcal P_\fun(H)$ and $H$ is Dedekind-finite. Therefore, we conclude from Proposition \ref{prop:funt&fun-have-the-same-system-of-lengths}\ref{it:prop:funt&fun-have-the-same-system-of-lengths(ii)} that $X \in \mathcal{A}(\mathcal P_\funt(H))$.
\end{proof}

\begin{thm}\label{prop:HF-exp-3}
	Let $H$ be a monoid. Then $\mathcal P_\fun(H)$ is \textup{HmF} if and only if $H$ is trivial or a cyclic group of order $3$.
	%	
	%	Then the following are equivalent:
	%	\begin{enumerate}[label={\rm (\alph{*})}]
	%	\item\label{it:prop:HF-exp-3(i)} $\mathcal P_\fun(H)$ is \textup{HmF}.
	%	\item\label{it:prop:HF-exp-3(ii)} $H$ is trivial or a cyclic group of order $3$.
	%	\end{enumerate}
\end{thm}

\begin{proof}
	The ``if'' part is an easy consequence of Theorem \ref{BmF-char} and Propositions \ref{prop:bounded-minimal-fzn}\ref{it:prop:bounded-minimal-fzn(i)} and \ref{prop:min-basics}\ref{it:prop:min-basics(i)}, when considering that, if $H$ is trivial or a cyclic group of order $3$, then $1_H \ne x^2 \ne x$ for all $x \in H \setminus \{1_H\}$ and every non-empty subset of $H$ has at most $3$ elements.
	%, and let $n \in \NN^+$ such that $1_H,x,\dots, x^n$ are pairwise distinct. We claim that $n \le 2$.
	
	
	As for the other direction, suppose $\mathcal P_\fun(H)$ is HmF and $H$ is non-trivial. Then $\mathcal P_\fun(H)$ is atomic, and we claim that $H$ is a $3$-group. By Theorem \ref{th:atomicity} and Lemma \ref{lem:no-non-id-elts-of-small-order-implies-structure}\ref{it:lem:no-non-id-elts-of-small-order-implies-structure(ii)}, it suffices to show that
	$
	x^3 \in \{1_H, x, x^2\}$ for every $x \in H$, since this in turn implies (by induction) that $\langle x \rangle_H \subseteq \{1_H, x, x^2\}$ and $\ord_H(x) \le 3$.
	
	For, assume to the contrary that $x^3 \notin \{1_H, x, x^3\}$ for some $x \in H$, and set $X := \{1_H, x, x^2, x^3\}$. By Theorem \ref{BmF-char}, $\mathfrak a := \{1_H, x\}^{\ast 3}$ and $\mathfrak b := \{1_H, x\} \ast \{1_H, x^2\}$ are both factorizations of $X$ in $\mathcal P_\fun(H)$; and in light of Proposition \ref{prop:min-basics}\ref{it:prop:min-basics(i)}, $\mathfrak b$ is in fact a minimal factorization (of length $2$). Then $\mathfrak a$ cannot be minimal, because $\mathcal P_\fun(H)$ is HmF and $\mathfrak a$ has length $3$. However, since $\mathcal P_\fun(H)$ is a reduced monoid (and $X$ is not an atom), this is only possible if $x^3 \in X = \{1_H, x\}^2$, a contradiction.
	
	So, $H$ is a $3$-group, and as such it has a non-trivial center $Z(H)$, see e.g. \cite[Theorem 2.11(i)]{gorenstein80}. Let $z$ be an element in $Z(H) \setminus \{1_H\}$, and suppose for a contradiction that $H$ is not cyclic. Then we can choose some element $y \in H \setminus \gen{z}_H$, and it follows from the above that $K := \gen{y,z}_H$ is an abelian subgroup of $H$ with $\ord_H(y) = \ord_H(z) = 3$ and $|K| = 9$. 
	We will prove that $K$ has $\preceq_{\mathcal P_\fun(H)}$-minimal factorizations of more than one length, which is a contradiction and finishes the proof.
	
	Indeed, we are guaranteed by Theorem \ref{BmF-char} that $\mathfrak c := \{1_H,y\}^{\ast 2} \ast \{1_H,z\}^{\ast 2}$ is a length-$4$ factorization of $K$ in $\P_\fun(H)$; and it is actually a minimal factorization, because removing one or more atoms from $\mathfrak c$ yields an $\mathcal{A}(\mathcal P_\fun(H))$-word whose image under $\pi_{\mathcal P_\fun(H)}$ has cardinality at most $8$ (whereas we have already noted that $|K| = 9$).
	On the other hand, it is not difficult to check that $A := \{1_H, y, z\}$ is an atom of $\mathcal P_\fun(H)$: If $\{1_H, y, z\} = YZ$ for some $Y, Z \in \mathcal P_\fun(H)$ with $|Y|, |Z| \ge 2$, then $Y, Z \subseteq \{1_H, y, z\}$ and $Y \cap Z = \{1_H\}$, whence $YZ = \{1_H, y\}\cdot\{1_H, z\} = K \ne A$. This in turn implies that $A^{\ast 2}$ is a length-$2$ factorization of $K$ in $\P_\fun(H)$, and it is minimal by Proposition \ref{prop:min-basics}\ref{it:prop:min-basics(i)}.
	So, we are done.
\end{proof}

\begin{cor}\label{cor:when-reduced-pm-is-minimally-factorial}
	Let $H$ be a monoid. Then $\mathcal P_\fun(H)$ is minimally factorial if and only if $H$ is trivial.
\end{cor}

\begin{proof}
	The ``if'' part is obvious. For the other direction, assume by way of contradiction that $\mathcal P_\fun(H)$ is minimally factorial but $H$ is non-trivial. Then $\mathcal P_\fun(H)$ is HmF, and we obtain from Theorem \ref{prop:HF-exp-3} that $H$ is a cyclic group of order $3$. Accordingly, let $x$ be a generator of $H$. By Lemma \ref{lem:2-elt-atoms}\ref{it:lem:2-elt-atoms(i)} and Proposition \ref{prop:min-basics}\ref{it:prop:min-basics(i)}, $\mathfrak a := \{1_H, x\}^{\ast 2}$ and $\mathfrak b := \{1_H, x^2\}^{\ast 2}$ are both minimal factorizations of $H$ in $\mathcal P_{\fin,1}(H)$. However, $(\mathfrak a, \mathfrak b) \notin \mathcal{C}_{\mathcal P_\fun}(H)$, because $\mathcal P_\fun(H)$ is a reduced monoid. Therefore, $\mathcal P_\fun(H)$ is not minimally factorial, so leading to a contradiction and completing the proof.
\end{proof}
%
At this point, we have completely characterized the correlation between the ground monoid $H$ and whether $\P_\fun(H)$ has factorization properties such as atomicity, BFness, etc., and their minimal counterparts.
In most cases, this extends to a characterization of whether the same properties hold for $\P_\funt(H)$, with the exception of the gap suggested by Theorem \ref{prop:HF-exp-3} and Corollary \ref{cor:when-reduced-pm-is-minimally-factorial}.
In particular, it still remains to determine the monoids $H$ which make $\P_\funt(H)$ HmF or minimally factorial.
However, what we have shown indicates, we believe, that the arithmetic of $\P_\fun(H)$ and $\P_\funt(H)$ is robust and ripe for more focused study.

\section{Cyclic monoids and interval length sets} \label{sec:finite-cyclic}
\label{sec:cyclic-case} 
For those monoids $H$ with $\P_\fun(H)$ atomic, we have by Proposition \ref{lem:no-non-id-elts-of-small-order-implies-structure} that the semigroup generated by an element $x\in H$ is isomorphic either to $\ZZ/n\ZZ$ or to $\NN$ under addition.
As such, we will concentrate throughout on factorizations in $\P_{\fin,0}(\\ZZ/n\\ZZ)$ and also mention some results on $\P_{\fin,0}(\NN)$ which are discussed in detail in \cite[\S{ }4]{fan-tringali18}.
At the end we will return to the general case, where the preceding discussion will culminate in a realization result (Theorem \ref{th:interval-lengths}) for sets of minimal lengths of $\P_\fun(H)$.

We invite the reader to review \S{ }\ref{subsec:generalities} before reading further. Also, note that, through the whole section, we have replaced the notation $\P_\fun(H)$ with $\P_{\fin,0}(H)$ when $H$ is written additively (cf. Example \ref{exa:strict-inclusion}).


\begin{defn}\label{NR-factorization}
	Let $X \in \mathcal P_{\fin,0}(\ZZ/n\ZZ)$. We say that a non-empty factorization $\mathfrak a = A_1 \ast \cdots \ast A_\ell \in \mathcal{Z}(X)$ is a \textit{non-reducible factorization} (or, shortly, an \textit{\textup{NR}-factorization}) if $ \max\hat{A}_1 + \dots + \max\hat{A}_\ell = \max \hat{X}$.
	%has \textit{no reduction $\fixed[-0.5]{\text{ }}{}\bmod n$} if $ \max\hat{A}_1 + \dots + \max\hat{A}_\ell = \max \hat{X}$, in which case we refer to $\mathfrak{a}$ as a
\end{defn}

This condition on factorizations will allow us to bring calculations up to the integers, where sumsets are more easily understood.
More importantly, NR-factorizations are very immediately relevant to our investigation of minimal factorizations.

\begin{lemma}\label{NR-factorizations-are-minimal}
	Any \textup{NR}-factorization in $\P_{\fin,0}(\\ZZ/n\\ZZ)$ is a minimal factorization.
\end{lemma}

\begin{proof}
	Let $\mathfrak{a} = A_1 \ast \cdots \ast A_\ell$ be an NR-factorization in $\mathcal P_{\fin,0}(\ZZ/n\ZZ)$ of length $\ell$, and assume for the sake of contradiction that $\mathfrak a$ is not minimal. Since
	$\P_{\fin,0}(\\ZZ/n\\ZZ)$ is reduced and commutative, the factorizations which are $\mathcal{C}_{\P_{\fin,0}(\\ZZ/n\\ZZ)}$-congruent to $\mathfrak{a}$ are exactly the words $A_{\sigma(1)} \ast \cdots \ast A_{\sigma(\ell)}$, where $\sigma$ is an arbitrary permutation of the interval $\llb 1, \ell \rrb$.
	So, on account of Proposition \ref{prop:min-basics}\ref{it:prop:min-basics(i)}, the non-minimality of $\mathfrak{a}$ implies without loss of generality that $\ell \ge 3$ and $
	X := A_1 + \cdots + A_\ell = A_1 + \cdots + A_k$ for some $k \in \llb 1, \ell-1 \rrb$. 
	
	Now, let $x \in X$ such that $\hat{x} = \max \hat{X}$. Using that $\mathfrak a$ is an NR-factorization, and considering that, for each $i \in \llb 1, \ell \rrb$, $A_i$ is an atom of $\mathcal P_{\fin,0}(\ZZ/n\ZZ)$ and hence $\max \hat{A}_i \ge 1$, it follows from the above that
	\begin{equation}\label{equ:inequ-with-maxima}
	\hat{x} = \max\hat{A}_1+\max\hat{A}_2+\cdots+\max\hat{A}_\ell > \max\hat{A}_1 + \dots + \max\hat{A}_k,
	\end{equation}
	On the other hand, since $X = A_1 + \cdots + A_k$, there are $a_1 \in A_1,\dots, a_k \in A_k$ such that $a_1+\dots+ a_k = x$, from which we see that $\hat{x} \equiv \hat{a}_1 + \cdots + \hat{a}_k \bmod n$. But it follows from \eqref{equ:inequ-with-maxima} that
	$
	0 \le \hat{a}_1+\dots+ \hat{a}_k < \hat{x} < n$, and this implies $\hat{x} \not\equiv \hat{a}_1 + \cdots + \hat{a}_k \bmod n$ (recall that, by definition, $\hat{X} \subseteq \llb 0, n-1 \rrb$). So we got a contradiction, showing that $\mathfrak a$ was minimal and completing the proof.
\end{proof}


We are aiming to find, for every $k\in\llb 2, n-1\rrb$, a set $X_k\in\P_{\fin,0}(\ZZ/n\ZZ)$ for which $\mathsf{L}^\m(X_k) = \llb 2, k \rrb$, on the assumption that $n\ge 5$ is odd:
Surprisingly, most of the difficulty lies in showing that $2\in \mathsf{L}^\m(X_k)$.
To do this, we first need to produce some large atoms.

\begin{prop}\label{large-atom-construction}
	Let $n\ge 5$ be odd.
	Then the following sets are atoms of $\mathcal P_{\fin,0}(\ZZ/n\ZZ)$:
	\begin{enumerate}[label={\rm (\roman{*})}]
		\item\label{it:large-atom-construction(i)} $B_h := \bigl\{\overline{0}\} \cup \{\overline{1},\overline{3},\dots, \overline{h}\bigr\}$ for odd $h\in \llb 1,(n-1)/2 \rrb$.
		\item\label{it:large-atom-construction(ii)} $C_1 := \bigl\{\overline{0}, \overline{2}\bigr\}$, $C_3 := \bigl\{\overline{0},\overline{2},\overline{3},\overline{4}\bigr\}$, and $C_\ell := B_\ell\cup\bigl\{\overline{\ell+1}\bigr\}$ for odd $\ell\in \llb 5, (n-1)/2 \rrb$.
	\end{enumerate}
\end{prop}

\begin{proof}
	\ref{it:large-atom-construction(i)} Let $h \in \llb 1, (n-1)/2 \rrb$ be odd, and suppose that $B_h = X + Y$ for some $X,Y\in \mathcal P_{\fin,0}(\ZZ/n\ZZ)$.
	Then $X$ and $Y$ are subsets of $B_h$, so
	\[
	\max\hat{X} + \max\hat{Y} \le 2\max\hat{B}_h = 2h \le n-1.
	\]
	Because $\overline{1}\in B_h$, we must have $\overline{1}\in X\cup Y$.
	However, if $\overline{1}\in X$ and $a\in Y$ for some $a \in B_h\setminus\{\overline{0}\}$, then  $1+\hat{a} \in \hat{X} + \hat{Y}$ is even, which is impossible since $\max\hat{X} + \max\hat{Y} < n$ and $\hat{B}_h \setminus\{0\}$ consists only of odd numbers.
	Thus $Y = \{\overline{0}\}$, and hence $B_h$ is an atom.
	
	\ref{it:large-atom-construction(ii)}
	$C_1$ is an atom by Lemma \ref{lem:2-elt-atoms}\ref{it:lem:2-elt-atoms(i)} and it is not too difficult to see that so is $C_3$. Therefore, let $\ell\ge 5$ and suppose $C_\ell = X+Y$ for some $X,Y\in \mathcal P_{\fin,0}(\ZZ/n\ZZ)$ with $X,Y\neq\bigl\{\overline{0}\bigr\}$.
	
	First assume that $\overline{\ell+1}\notin X\cup Y$. Then $\hat{X}$ and $\hat{Y}$ consist only of odd integers, so $\hat{x}+\hat{y}$ is an even integer in the interval $\llb 2, n-1 \rrb$ for all $x\in X\setminus\bigl\{\overline{0}\bigr\}$ and $y\in Y\setminus\bigl\{\overline{0}\bigr\}$.
	However, $\hat{X}+\hat{Y} = \hat{C}_\ell$ and the only non-zero even element of $\hat{C}_\ell$ is $\ell+1$. Thus, it must be that $X = \bigl\{\overline{0},x\bigr\}$ and $Y = \bigl\{\overline{0},y\bigr\}$ for some non-zero $x, y \in \ZZ/n\ZZ$, with the result that $|X+Y| \le 4 < |C_\ell|$, a contradiction.
	
	It follows (without loss of generality) that $\overline{\ell+1}\in Y$.
	Then $X \subseteq \bigl\{\overline{0},\overline{\ell},\overline{\ell+1}\bigr\}$, for, if $x\in X$ with $0 < \hat{x} < \ell$, then $\hat{x} + \ell+1 \in \hat{C}_\ell$, which is impossible since $\hat{x}+\ell+1 \in \llb \max\hat{C}_\ell+1 , n-1 \rrb$.
	This in turn implies that $Y \subseteq \bigl\{\overline{0},\overline{1},\overline{\ell},\overline{\ell+1}\bigr\}$ for similar reasons. As a consequence,
	\[
	X+Y
	\subseteq \bigl\{\overline{0},\overline{\ell},\overline{\ell+1}\bigr\}
	+ \bigl\{\overline{0},\overline{1},\overline{\ell},\overline{\ell+1}\bigr\}
	= \bigl\{ \overline{0}, \overline{1}, \overline{\ell}, \overline{\ell+1}, \overline{2\ell},\overline{2\ell+1}, \overline{2\ell+2} \bigr\}
	\]
	However, $\ell+1 < 2\ell \le n-1$, so we cannot have $\overline{2\ell}\in X+Y$. Then $2\ell+1 = n$, in which case $\overline{2\ell+1} = \overline{0}$ and $\overline{2\ell+2} = \overline{1}$; or $2\ell+1 < n$, so that $\overline{2\ell+1}, \overline{2\ell+2} \notin C_h$ (recall that $\ell \le (n-1)/2$). In either case, we get $X+Y \subseteq \bigl\{ \overline{0}, \overline{1}, \overline{\ell}, \overline{\ell+1}\bigr\}$, hence $|X+Y| \le 4 < |C_\ell|$, which is a contradiction and leads us to conclude that $C_\ell$ is an atom.
\end{proof}

Now that we have found large atoms in $\P_{\fin,0}(\ZZ/n\ZZ)$, we can explicitly give, for each $k\in \llb 2,n-1\rrb$, an element $X_k\in \P_{\fin,0}(\ZZ/n\ZZ)$ which has a (minimal) factorization of length $2$.

\begin{lemma}\label{2-atom-factorization}
	Fix an odd integer $n\ge 5$ and let $k\in \llb 2, n-1 \rrb$.
	Then the set $X_k = \{\overline{0},\overline{1},\dots, \overline{k}\}$ has an \textup{NR}-factorization into two atoms in $\P_{\fin,0}(\\ZZ/n\\ZZ)$.
\end{lemma}

\begin{proof}
	We will use the atoms $B_h$ and $C_\ell$ as defined in Proposition \ref{large-atom-construction}. We claim that, for every $r\in\{0,1\}$ and all odd $h\in \llb 1, (n-1)/2 \rrb$,
	%
	\[
	\hat{B}_{h+2r}+\hat{C}_h = \llb 0, 2h+2r+1 \rrb
	\quad\text{and}\quad
	\hat{C}_{h+2r} + \hat{C}_{h} = \llb 0, 2r+2h+2\rrb.
	\]
	We will only demonstrate that $\hat{B}_h + \hat{C}_h = \llb 0, 2h+1 \rrb$ (the other cases are an easy consequence).
	The claim is trivial if $h=1$ or $h=3$, so suppose $h \ge 5$.
	Then
	\[
	\hat{B}_h + \hat{C}_h \supseteq \{1,3,\dots,h\} + \{0, h+1\} = \{1,3,\dots, 2h+1\}
	\]
	and
	\[
	\hat{B}_h + \hat{C}_h \supseteq \{1,3,\dots,h\} + \{1, h\} = \{2,4,\dots, 2h\},
	\]
	so $\hat{B}_h +\hat{C}_h \supseteq \llb 0 , 2h+1 \rrb$.
	This gives that $\hat{B}_h +\hat{C}_h = \llb 0 , 2h+1 \rrb$, since $\max\hat{B}_h+\max\hat{C}_h = h+(h+1)$.
	
	Accordingly, we now prove that $X_k$ can be expressed as a two-term sum involving $B_h$ and $C_\ell$, for some suitable choices of $h$ and $\ell$ depending on the parity of $k$.
	\begin{enumerate}[leftmargin=1.8cm,label={\textsc{Case }\arabic{*}:}]
		\item $k = 2m+1$ (i.e., $k$ is odd). Then it is immediate to verify that $X_k = B_m+C_m$ if $m$ is odd, and $X_k = B_{m+1} +C_{m-1}$ if $m$ is even.
		\item $k = 2m$ (i.e., $k$ is even). Since $X_2 = B_1+B_1$ and $X_4 = B_1+B_3$, we may assume $m\ge3$. Then it is seen that $X_k = C_{m} + C_{m-2}$ if $m$ is odd, and $X_k = C_{m-1}+C_{m-1}$ if $m$ is even.
	\end{enumerate}
	We are left to show that the decompositions given above do in fact correspond to minimal factorizations. As an example, consider the case when $k=2m+1$ and $m$ is odd (the computation will be essentially identical in the other cases).
	Then $\max\hat{B}_{m}+\max\hat{C}_m = 2m+1$, so that $B_m \ast C_m$ is an NR-factorization of $X_k$, and is hence minimal by Proposition \ref{NR-factorizations-are-minimal}.
\end{proof}



\begin{lemma}\label{lem:interval-minimal-length-sets}
	Fix an odd integer $n\ge 3$ and, for each $k\in \llb2,n-1\rrb$, let $X_k := \{\overline{0},\overline{1}, \dots, \overline{k} \} \in  \P_{\fin,0}(\ZZ/n\ZZ)$.
	Then $\mathsf{L}^\m(X_k) = \llb 2,k \rrb$.
\end{lemma}

\begin{proof}
	We have already established in Lemma \ref{2-atom-factorization} that $X_2$ has an NR-factorization of length $2$.
	Now fix $k \in \llb 3, n-1 \rrb$ and suppose that, for all $h \in \llb 2, k-1 \rrb$ and $\ell \in \llb 2,h \rrb$, $X_h$ has an NR-factorization of length $\ell$.
	Choose some $\ell \in \llb 2,k-1 \rrb$; $X_{k-1}$ has an NR-factorization $\mathfrak{a}$, and it is straightforward to see that $\{\overline{0},\overline{1}\}*\mathfrak{a}$ is an NR-factorization of $X_k$.
	Letting $\ell$ range over $\llb 2,k-1\rrb$, this argument, Lemma \ref{NR-factorizations-are-minimal}, and Lemma \ref{2-atom-factorization} imply that $\mathsf{L}^\m(X_k) \supseteq \llb 2, k \rrb$.
	Moreover, Proposition \ref{prop:bounded-minimal-fzn}\ref{it:prop:bounded-minimal-fzn(i)} yields the other inclusion and so we have  $\mathsf{L}^\m(X_k) = \llb 2,k \rrb$.
\end{proof}


\begin{lemma}\label{prop:intervals-in-N}
	Let $H$ be a non-torsion monoid.
	Then $\mathcal{L}(\P_{\fin,0}(\NN)) \subseteq \mathcal{L}^\m(\P_\fun(H))$, and for every $k\ge 2$ there exists $Y_k\in \P_\fun(H)$ with $\mathsf{L}^\m(Y_k) = \llb 2, k \rrb$.
\end{lemma}

\begin{proof}
	Suppose that $y\in H$ has infinite order, and set $Y := \{y^k: k \in \NN\}$. Clearly, $Y$ is a submonoid of $H$, and the (monoid) homomorphism $(\NN,+) \to Y: k \mapsto y^k$ determined by sending $1$ to $y$ induces an iso\-morphism $\P_{\fin,0}(\NN) \to \P_\fun(Y)$.
	Since, by Proposition \ref{prop:pm-arith}\ref{it:prop:pm-arith(iii)}, $\P_\fun(Y)$ is a divisor-closed submonoid of $\P_\fun(H)$, we thus have by parts \ref{it:prop:min-basics(iii)} and \ref{it:prop:min-basics(iv)} of Proposition \ref{prop:min-basics} that
	\[
	\mathcal{L}(\P_{\fin,0}(\NN)) = \mathcal{L}^\m(\P_{\fin,0}(\NN)) = \mathcal{L}^\m(\P_\fun(Y)) \subseteq \mathcal{L}^\m(\P_\fun(H)).
	\]
	The rest of the statement now follows from the above and \cite[Proposition 4.8]{fan-tringali18}.
\end{proof}




\begin{thm}\label{th:interval-lengths}
	Assume $H$ is a monoid such that $1_H \neq x^2 \neq x$ for all $x\in H\setminus\{1_H\}$, and set $N := \sup\{\ord_H(x) : x\in H \}$.
	Then $\llb 2, k \rrb \in \mathcal{L}^\m(\P_\fun(H))$ for every $k \in \llb 2, N-1 \rrb$.
\end{thm}

\begin{proof}
	If $H$ is non-torsion, this follows immediately from Lemma \ref{prop:intervals-in-N}.
	Otherwise, let $k\in \llb 2, N-1 \rrb$ and $y\in H$ with $n := \ord_H(x) > k$.
	Then $Y :=\gen{y}_H \cong \ZZ/n\ZZ$, so we have by Proposition \ref{prop:pm-arith}\ref{it:prop:pm-arith(iii)}, Lemma \ref{lem:interval-minimal-length-sets}, and Proposition \ref{prop:min-basics}\ref{it:prop:min-basics(iii)} that
	$\llb 2, k \rrb \in \mathcal{L}^\m(\P_\fun(Y)) \subseteq \mathcal{L}^\m(\P_\fun(H))$.
\end{proof}
